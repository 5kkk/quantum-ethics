\chapter*{History of this book}
\addcontentsline{toc}{chapter}{\protect History of this book}

\renewcommand{\epigraphwidth}{8.3cm}
\epigraph{For one of the most stringent tests of any physical theory is the prediction of its own creation process.}{Sébastien Fauvel,\\\textit{Quantum Ethics}~\cite{Fauvel2013}}

How would Quantum Field Theory look like if we stopped for a while developing it further as if it were the draft of a yet-to-be-discovered Theory of Everything, and just started to reformulate the Standard Model as a mathematically and conceptually coherent physical theory? And what would such a theory tell us about the world and about ourselves, which remains hidden in the ill-defined formulations we've grown up with through the last decades? As I started back in 2010 to reflect on these questions, I didn't have yet a clear vision of what this work would lead me to.
I just had the feeling that these very basic questions hadn't been interesting anyone any more for a far too long time, and that we should actually have the means by now, with our understanding of Renormalization, of writing down a well-defined Quantum Field Theory reasonably accounting for all known experimental data (excepted General Relativity phenomena) -- which means essentially that it has to be compatible with the Standard Model at known energy scales.
I was quite confident that I could find a physically sound regularization of the Standard Model, which I simply wouldn't consider as an approximation, but take as the exact theory itself, the Standard Model being an ill-defined idealization of it.
The models used in computer simulations of lattice Quantum Chromodynamics, for instance, would show me the way.
Of course, I knew that I wouldn't be able to derive the theory from the usual first principles any more, but given that all the attempts of axiomatic Quantum Field Theory to construct well-defined interacting fields upon these first principles had failed miserably, I thought that maybe they could be misleading in the end.
Anyhow, I had never been very fond of the heuristical construction of Quantum Field Theory based on Gauge and Poincaré invariance.
Developing the whole mathematical apparatus of Representation Theory to simply derive the expression of spin 1 and spin 1/2 spinors as irreducible unitary representations of the Poincaré group had always seemed far too expensive to me, and Gauge transformations mixing particle fields far too artificial to make up a fundamental symmetry of Nature.

So I felt free to redefine the Hilbert space of the quantum states without paying much attention to these first principles and focused instead on the mathematical well-definedness of the theory, and in particular of the Schrödinger equation.
The most evident way of insuring a well-defined solution at all times is to make the Hilbert space finite dimensional, which has two major physical implications.
The most important one is that the physical space itself, too, has to be finite, \textit{i.e.} to consist in a finite number of points.
The simplest way to take this constraint into account is to define space as a finite lattice, like in computer simulations of Quantum Chromodynamics, and to adapt the expression of the Hamiltonian operator of the Standard Model, developed on the momentum basis, by simply using a discrete Fourier transform on the lattice.
This formulation of the theory, considered as a fundamental theory and not as a numerical approximation, has evident ontological and cosmological implications.
It is interesting to see, for instance, how modern physics addresses thus the atomist polemic of ancient Greece, \textit{i.e.} the question if matter can be, in principle, indefinitely separated into smaller pieces, or if there are smallest building blocks of matter.
The answer of this theory were not only that elementary particles are the smallest, point-like building blocks of matter, but that space itself is constituted of smallest, point-like building blocks, and even of a finite number of them! Incidentally, the void between point-like particles imagined by Greek atomists like Democritus acquires a very different quality, too.
There is still a notion of void as the unrealized potentiality of the presence of matter, represented by an unoccupied lattice site, but this site, although it is empty of matter, is still something familiar, identifiable, something we could put a name on.
Psychologically, the void loses thus much of the threatening quality of the indiscernible.
The empty space that we might tend to imagine between the lattice sites isn't actually part of the material world, it is purely virtual and has no physical relevance.

From a cosmological perspective, the finiteness of space is also a very interesting aspect.
It addresses the old question of knowing whether there is something like a frontier of the universe or if the universe is infinite, and it offers a very original answer.
According to this theory, the universe is both finite and boundless; it actually has a toroidal structure, which is not of topological nature, but reveals itself at the level of the field dynamics: Wave packets will transit smoothly from one side of the finite lattice to the opposite one without experiencing any discontinuity.
So the light we emit, for instance, could come back to us from the opposite direction after having traveled through the whole universe.
Yes, if the universe were smaller, maybe you could see the Earth looking at the stars...
and the position of the closest images of the Earth in the night sky would give you the direction of the lattice axes, by the way.

The second physical implication of the finite dimension of the Hilbert space is the existence of a maximum occupation number for boson fields.
I wondered if there were any good theoretical reason to assume an unbounded number of bosons per field mode, and I actually didn't find any.
Of course, the commutation relations usually considered as essential properties of the creation operators would break down when the maximum number of particles is being reached, but these relations, relicts of a heuristical construction of Quantum Field Theory based on the harmonic oscillator model of Quantum Mechanics, are not really necessary to define creation operators.
In fact, it is quite straight-forward to define a basis of the Hilbert space on a finite lattice, you just have to take as basis vectors field configurations defined as functions giving the number of particles of each kind at each lattice site.
And it isn't more complicated either to define creation operators as adding one particle of a given kind at a given lattice site, as long as a given maximum occupation number hasn't been reached.
The normalization factors implied by the commutation relations can then be moved to the spinors, where they actually belong.
The situation is quite similar for fermions: If you don't construct the Hilbert space heuristically as a Fock space over the one-particle Hilbert space of Quantum Mechanics, the sign factors implied by the anticommutation of the creation operators can be moved to the spinors too.
So in the end, there isn't any qualitative distinction to be made between bosons and fermions; the same creation operators can be used in both cases, differing only in their maximum occupation numbers.
In fact, if you don't construct the Hilbert space as a Fock space, but define it directly (or use a Fock space modulo particle labels permutations), there is no Spin-Statistics Theorem classifying particles into bosons and fermions according to their spin any more.
This famous theorem relates the integer or half-integer character of the spin to the possible sign change happening to the quantum state when the labels of two particles of the same type are being exchanged.
But the notion of exchanging the labels of two particles doesn't actually have any physical meaning, it only makes sense in the Fock space formalism, and is a mere mathematical artifact.
I think it is important to realize that the Spin-Statistics Theorem, traditionally considered as one of the greatest insights provided by Special Relativity into Quantum Field Theory, actually doesn't have any profound physical meaning, and doesn't establish, as it is often being stated, a connexion between the geometry of space-time and the collective behavior of particles.
It only expresses a property of the ``unphysical'' Fock space formalism, and becomes meaningless as soon as you consider the ``physical'' quantum states modulo particle labels permutations.
So the categories of `bosons' and `fermions' are not implied by Special Relativity, as far as their collective behavior is concerned; only the form of the spinors is.
Determining experimentally the maximum occupation number for each boson field is still an open question: For ``heavy'' bosons like the Z boson, for instance, I don't think that a lower bound much greater that one can already be established with current experimental data...

Once I had constructed this well-defined framework for Quantum Field Theory and made a first proof-of-concept by integrating Quantum Electrodynamics, I left the paper draft I had written by that time rest for a while, took care of my new-born son and started reading a book from the Philosophy library of my wife that had been intriguing me for a while: A French translation of Spinoza's \textit{Ethics}.
The reading would accompany me through the whole summer of 2011 and make a lasting impression on me.
The subtle way Spinoza integrates subjective experience into the physical world reminded me of von Neumann's hypothesis that mind could somehow cause the collapse of the quantum state of a system upon measurement, and I realized that, within the well-defined framework I had constructed, we had the possibility for the first time to give a formally very precise definition of what von Neumann had meant.
This would provide a precise answer to the measurement problem, and probably the first one that isn't only psychologically motivated, but also constrained by formal consistency.
So I started to figure out how to relate subjective experience to the state of the material world in quantum physical terms, and re-read Spinoza with this question in mind.
Following von Neumann's interpretation, I should relate a mental state to a Hilbert subspace in such a way that the Hilbert space be a direct sum of the subspaces corresponding to each possible mental state.
Making the assumption that we have to do with different states of a single subjective experience in this decomposition leads directly to the paradox of Wigner's friend, that is, when several bodies (brains?) are present at once -- and it is the case most of the time, isn't it? --, which one oughts to determine the mental state and trigger the collapse? Escaping this issue requires to describe the mental state in its totality, \textit{i.e.} to specify the number of subjects having each possible subjective experience at a time, so that a mental state is, basically, described with the same formalism as a field configuration over subjective experiences.
And exactly as this is the case for particles in particle fields, subjects are \textit{indistinguishable} at a fundamental level.
There is nothing like ``my'' mind or ``your'' mind, each one having its own personal history that could, in principle, be tracked back from birth to death.
Pretty much like single particles don't have any individual trajectory in Quantum Physics, single subjects don't have any individual history either.
As Spinoza would say, we are all thinking together in God; we participate of a single mental reality and don't have any individual existence below this ontological level.
This will probably sound crazy to most readers, and it is probably one of the reasons why Spinoza has been excommunicated for heresy in his time.
But it is actually an utmost self-consistent point of view, and the only one consistent with Quantum Field Theory so far.
I cannot but warmly advise you to take a closer look at \textit{The Ethics}; re-reading Spinoza and seeing how a 17th century heresy meets Quantum Physics is really a very exciting experience.
The pantheist thesis of Spinoza fits incredibly well in the world view sustained by Quantum Field Theory; neither your body, enmeshed by quantum entanglement with other ones, nor your mind, indistinguishable from other ones, have any individual existence: Nothing exists but God, aka Nature.
This is basically the idea of this book, and given that no other interpretation of Quantum Physics integrates so deeply into the formalism of Quantum Field Theory, this made me think that this book was worth writing it, and I guess it will be a joy for many science philosophers to see that the latest achievements in fundamental physics are leading us back, eventually, from a materialistic to a pantheist philosophy.

As soon as I had developed this Spinozist model of the mental world (which builds up, together with the material world of quantum fields, the physical world as a whole), I got confronted with the old question of the status of time in Quantum Physics.
The controversies on this subject have been summarized very concisely by Wolfgang Pauli in his statement that there cannot be any time observable in Quantum Physics.
In the Copenhagen interpretation, indeed, time isn't a property of the quantum system under observation; it isn't being measured quantum physically, but \textit{classically}, and correlated with quantum measurement results.
When you measure the fluorescence lifetime of ruby, for instance, you only measure the presence of emitted photons on a quantum physical way, which implies the collapse of the system's quantum state, but you measure the time at which the photodetector gets activated by simultaneously reading a clock in a classical way.
That is a very strange feature of the quantum/classical dichotomy of the Copenhagen interpretation, and it leaves one very basic question completely open: There is no way to predict quantum physically \textit{when} the quantum measurement process and the collapse of the quantum state will take place, or even to find out the time distribution of the measurement process in a statistical way.
The Copenhagen interpretation only defines the statistical distribution of the possible measurement results assuming a measurement is being performed at a given time, but doesn't tell anything about the conditions under which a quantum measurement will actually happen -- basically because measuring is considered as an act taking place in the classical world, which escapes quantum physical description.
The reason why this uncertainty about the time at which a quantum measurement happens doesn't have any consequences on our ability to derive statistical results from the theory was already clear in the 1930's: As von Neumann pointed out, it wouldn't make any statistical difference if the collapse of the quantum state happened upon an interaction of the quantum system with a measurement apparatus, or upon an interaction of the quantum system including measurement apparatus with the observer, or at any stage inbetween.
And even if the observer wouldn't read the output of the measurement apparatus, the interaction of the quantum system with it, like any process introducing a strong correlation of its state with the environment, would yield quantum decoherence effects which are practically impossible to tell apart from the effects of an hypothetical collapse, as far as the statistical measurement results are concerned.
So we have practically no means of finding out at which stage the collapse is taking place, and addressing this question remains a purely theoretical issue of no practical interest.
Nevertheless, it has to be addressed by any theory going beyond the Copenhagen interpretation and trying to describe collapse as a physical process independent of the free will of the observer, which is subsumed in the classical world view.
There are lots of so-called spontaneous collapse theories, developed originally by John Bell followed by many others, which generally describe collapse as a dynamical process, yielding \textit{in fine} to the same states as an abrupt orthogonal projection would do.
But these models are purely materialistic and don't address the question of describing subjective experience in physical terms.
The suggestion of von Neumann that mind could cause the collapse of the quantum state, which would get projected to quantum states of the brain corresponding to a definite subjective experience, seemed much more promising to me, as I was looking forward to sketching a more comprehensive world view in physics.
So I stuck to the rather conservative hypothesis of an abrupt collapse of the quantum state via a random orthogonal projection to one of the Hilbert subspaces corresponding to a given mental state, and I had to define precisely when this process would happen.
In doing so, you are totally free as a theoretical physicist, because, as I said before, collapse and quantum decoherence have practically the same signature in statistical measurement results, so that we can never be sure of having observed a collapse or not.
I rejected the hypothesis of a continuous collapse, because continuous stochastic processes are only idealizations, so I supposed rather that collapses happen at discrete times.
This implies that our mental state evolves discontinuously, although we usually don't notice it.
From a phenomenological point of view, this isn't very surprising: Our impression of continuity is based on short-time memory and intentionality, not on the permanence and continuity of our subjective experience itself.
Even if we had a single, isolated mental experience, it would have the same quality and provide the same sensation of time as a continuous one -- for as the poet says, eternity lies in every moment...
The continuity of time only applies at the material level, while the mental world only picks out single ``snapshots'' of the state of the material world, so to say.
Determining when these mental experiences take place cannot be achieved by investigating their subjective content alone; only the elusive effects of the simultaneous collapse of the quantum state could indicate this.
So for the sake of simplicity, I just assumed a periodic collapse with a given elementary period, in order to have a well-defined model, even if we don't have yet any experimental clues in this respect.
Of course, the collapse of the quantum state is not a local process in the sense of Relativity Theory, but Einstein-Podolsky-Rosen experiments have already shown very clearly that this non-locality is really part of Nature.
And after all, who would expect mental phenomena to be local? They are not bounded to their material substrate; they don't live in the frame of space, but in another dimension of the physical world, so to say.

In the end, the model I'm proposing can be roughly described in very simple terms: A mental state is being experienced while the quantum state is undergoing an elementary unitary evolution, then a new mental state is being randomly moved to as the quantum state gets projected to the corresponding subspace, an so on.
In the meanwhile, this almost sounds trivial to me, so I guess I'm eventually understanding Quantum Physics, at least in this form.
This alone would be a revolution in this field of science.
But I'm not interested in pretending to have discovered deep truths about ``the inmost force which binds the world'', to speak with Goethe; I just wanted to show that it is possible, and actually quite easy, to give Quantum Field Theory a form and an interpretation which make it a formally and conceptually closed theory, capable of giving a well-defined answer to any question we can ask it -- even if we may eventually find out that it wasn't the right one.
This interpretation challenges all existing ones insofar as it is the first time that this degree of conceptual precision and formal well-definedness has been reached, and I hope this will be motivation enough for others to work out alternative interpretations and achieve the same level of quality -- so that we can finally know what Quantum Theory is actually about...
