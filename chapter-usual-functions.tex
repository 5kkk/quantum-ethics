\chapter{Usual functions}

\section{The sinc function}
\label{sinc}

In this document, the sinc function is defined by:
\begin{equation*}
\sinc{X} \eqdef \begin{cases}
1 & \text{for $X = 0$} \\
\sin{\pi X} / ( \pi X ) & \text{otherwise}
\end{cases}
\end{equation*}
This function admits following integral expression:
\begin{equation*}
\sinc{X} = \frac 1 X \int_{-X/2}^{X/2} \exp{\i 2 \pi x} \dx
\end{equation*}
and is normalized by:
\begin{equation*}
\int_{-\infty}^{+\infty} \sinc{X} \dX = 1
\end{equation*}

\section{The esinc function}
\label{esinc}

In this document, the esinc function is defined by:
\begin{equation*}
\esinc{X} \eqdef \exp{\i \pi X} \sinc{X}
\end{equation*}
where the sinc function is defined as in appendix \ref{sinc}.
This function admits following integral expression:
\begin{equation*}
\esinc{X} = \frac 1 X \int_0^X \exp{\i 2 \pi x} \dx
\end{equation*}
can be written:
\begin{equation*}
\esinc{X} = \frac{\sin{2 \pi X}}{2 \pi X} + \i \frac{1 - \cos{2 \pi X}}{2 \pi X}
\end{equation*}
and verifies:
\begin{equation*}
\esinc{-X} = \cc{\esinc{X}}
\end{equation*}

\section{Nascent delta functions}
\label{delta}

In this document, we make use of following nascent delta functions, which converge to the delta energy distribution for $t - t_0 \to \infty$:
\begin{eqnarray*}
\deltaE1{E} & \eqdef & \frac{t - t_0} \h \sinc{\frac{t - t_0} \h E} \\
\deltaE2{E} & \eqdef & \frac{t - t_0} \h \sinc{\frac{t - t_0} \h E}^2
\end{eqnarray*}
where the sinc function is defined as in appendix \ref{sinc}.
The square of the first one can be expressed in terms of the second one as:
\begin{equation*}
\deltaE1{E}^2 = \frac{t - t_0} \h \deltaE2{E}
\end{equation*}

We also make use of following family of functions converging to a distribution as $t - t_0 \to \infty$:
\begin{equation*}
\deltaE\PV{E} \eqdef 2 \frac{t - t_0} \h \esinc{\frac{t - t_0} \h E}
\end{equation*}
where the esinc function is defined as in appendix \ref{esinc}.
Its limit is given by:
\begin{equation*}
\lim_{t - t_0 \to \infty} \deltaE\PV{E} = \delta(E) + \frac \i \pi \PV \left( \frac 1 E \right)
\end{equation*}
where the Cauchy principal value of $1/E$ is defined by its action on any test function $\phi(E)$ by:
\begin{eqnarray*}
\left( \PV \left( \frac 1 E \right), \phi(E) \right) & \eqdef & \PV \int_{-\infty}^{+\infty} \frac{\phi(E)}E \dE \\
& = & \lim_{\varepsilon \to 0^+} \left( \int_{-\infty}^{-\varepsilon} \frac{\phi(E)}E \dE + \int_{\varepsilon}^{+\infty} \frac{\phi(E)}E \dE \right)
\end{eqnarray*}
