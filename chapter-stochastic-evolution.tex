\chapter{Stochastic evolution}
\label{Stochastic evolution}

\renewcommand{\epigraphwidth}{7cm}
\epigraph{It sounded quite a sensible voice, but it just said, ``Two to the power of one hundred thousand to one against and falling,'' and that was all.}{Douglas Adams,\\\textit{The Hitchhiker's Guide to the Galaxy}~\cite{Adams1979}}

\section{Collapse and collective mind selection}

In the joint evolution of the mental and quantum states of the universe, I suppose that the quantum state $\ket \Psi$ periodically becomes randomly projected into one of the subspaces $\H_\cms$, corresponding to a given mental state $\cms$, with a probability given by:
\begin{equation*}
P(\cms) = \frac{\bra\Psi \op{\Pi}_\cms \ket\Psi}{\braket\Psi\Psi}
\end{equation*}
where $\op{\Pi}_\cms$ is the orthogonal projection operator on $\H_\cms$. Furthermore, I suppose that this projection corresponds to the fact that, in the mental world, the collective mind state $\cms = \bFM N \ims$ is being experienced, i.e. each possible individual mind state $\ims$ is being experienced by $\bM N \ims$ individual minds. We call the material part of this process ``collapse'' of the quantum state of the universe and its mental counterpart ``collective mind selection''. The operators $\op{\Pi}_\cms$ are called ``collapse operators''. As a working hypothesis, we assume that the period $\tau$ of this process is of the order of the Plank time:
\begin{equation*}
\tau \approx \sqrt{4\pi\G \h/\c^5}\approx4.8~10^{-43}\ \mathrm s
\end{equation*}

\section{Mental evolution}

Fundamentally, Quantum Field Theory also defines the probability that any given succession of collective mind states be experienced, an initial quantum state being given. In the case where the actually experienced collective mind state has a relatively high probability, our mental state may give us some clues about the physics of the world we live in; on the opposite, if our collective mind state has a very low probability, our mental experience has very little to do with the laws of the physical world and we live in a mere illusion of knowing something about the material reality -- without having any mean of noticing it. This dilemma is very well known of particle physicists, who have to accept they cannot make more precise statements about reality than, for instance, ``in the context of the standard model and in the presence of a sequential fourth family of fermions with high masses [...] a Higgs boson with mass between 144 and 207 GeV/$\mathrm{c}^2$ is ruled out at 95\% confidence level''~\cite{CMS2011}. Any physical model can also be conventionally, but not definitely, ``ruled out'' if it predicts the observed results with a probability considered to be too low.

Given an initial quantum state $\ket{\Psi_0} \in \H_{\cms_0}$ at $t = 0$, the probability $P(\cms_t,t)$ that a given collective mind state $\cms_t$ is being realized at time $t \tau$, where $t \in \IN^*$, reads for $t = 1$:
\begin{equation*}
P(\cms_1,1) = \bra{\Psi_0} \dop U_\tau \op \Pi_{\cms_1} \op U_\tau \ket{\Psi_0} / \braket{\Psi_0}{\Psi_0}
\end{equation*}
where $\op U_\tau \eqdef \exp{- \i 2 \pi \Hop \tau / \h}$, and more generally for $t \geq 2$:
\begin{equation*}
P(\cms_t,t) = \sum_{\cms_{t - 1}} \dotsm \sum_{\cms_1} \bra{\Psi_0} \dop U_\tau \op \Pi_{\cms_1} \dotsm \dop U_\tau \op \Pi_{\cms_t} \op U_\tau \dotsm \op \Pi_{\cms_1} \op U_\tau \ket{\Psi_0} / \braket{\Psi_0}{\Psi_0}
\end{equation*}
If the initial vector state isn't exactly known, averaging on an orthonormal basis of $\H_{\cms_0}$ leads to:
\begin{equation*}
\left< P(\cms_t,t) \right>_{\cms_0} = \sum_{\cms_{t - 1}} \dotsm \sum_{\cms_1} \mathrm{Tr}_{\H_{\cms_0}} \dop U_\tau \op \Pi_{\cms_1} \dotsm \dop U_\tau \op \Pi_{\cms_t} \op U_\tau \dotsm \op \Pi_{\cms_1} \op U_\tau / \mathrm{dim} \H_{\cms_0}
\end{equation*}

\section{Transition probability}

We consider, to simplify the discussion, a repeated experiment with a single possible outcome, which may have been realized or not after a given duration $t \tau$. Notice that this duration doesn't correspond to the instant at which some physical event occurs, but is a sufficiently long duration after which the experimenter can consciously remember of having (just) observed the expected outcome or not.

The possible states of the minds corresponding to the beginning of the experiment are written $\cms_i$ and the initial state of the quantum fields is also an element of the Hilbert subspace $\H_i$ given by:
\begin{equation*}
\H_i = \bigoplus_{\cms_i}^\perp \H_{\cms_i}
\end{equation*}
The possible states of the minds corresponding to the measurement of the given outcome resp. of its absence are written $\cms_f^+$ resp. $\cms_f^-$. If the experiment works correctly, the final state of the quantum fields is, after measurement, an element of either of the Hilbert subspaces $\H_f^+$ or $\H_f^-$ given by:
\begin{equation*}
\H_f^\pm = \bigoplus_{\cms_f^\pm}^\perp \H_{\cms_f^\pm}
\end{equation*}
If the experiment fails for some reason (e.g. if some measuring device is getting damaged during the experiment), the final state of the quantum fields is orthogonal to $\H_f^+ \oplus \H_f^-$.

The absolute probability of measuring the given outcome resp. its absence is given by:
\begin{equation*}
\TP{\H_i}{\H_f^\pm} = \sum_{\cms_{t - 1}} \dotsm \sum_{\cms_1} \mathrm{Tr}_{\H_i} \dop U_\tau \op \Pi_{\cms_1} \dotsm \dop U_\tau \op \Pi_{f^\pm} \op U_\tau \dotsm \op \Pi_{\cms_1} \op U_\tau / \mathrm{dim} \H_i
\end{equation*}
where $\op \Pi_{f^\pm} = \sum_{\cms_f^\pm} \op \Pi_{\cms_f^\pm}$. The conditional probability of measuring the given outcome if the experiment doesn't fail is then given by:
\begin{equation*}
\mathcal T \TP{\H_i}{\H_f^+} = \frac{\TP{\H_i}{\H_f^+}}{\TP{\H_i}{\H_f^+} + \TP{\H_i}{\H_f^-}}
\end{equation*}
and we call it ``transition probability'' from $\H_i$ to $\H_f^+$.

If the experiment is conceived in such a way that the studied system is isolated from the observer for the duration of the experiment until it interacts with some measurement apparatus, the experiment is considered to have failed if the observer has gained some information about the studied system before it interacts with this apparatus. An intermediate observation of the system, as it would leave a permanent trace in the memory of the observer, would lead with a vanishingly small probability to a final state of the minds in which the observer isn't conscious of having made this observation. The only intermediate states of the minds $\cms_1, \dotsc, \cms_{t - 1}$ to be considered in the above sums (i.e. which haven't a vanishingly small contribution to the transition probability) correspond also to projectors that don't affect the Hamiltonian evolution of the studied system. In that case, the absolute probability of measuring the given outcome resp. its absence can be approximated by:
\begin{equation*}
\TP{\H_i}{\H_f^\pm} \approx \mathrm{Tr}_{\H_i} \dop U_{t \tau} \op \Pi_{f^\pm} \op U_{t \tau} / \mathrm{dim} \H_i
\end{equation*}
and can be written as a sum resp. a mean on quantum states forming an orthonormal basis of $\H_f^\pm$ resp. $\H_i$:
\begin{eqnarray*}
\TP{\H_i}{\H_f^\pm} & \approx & \sum_f \left< \TP i f \right>_i \\
\TP i f & \eqdef & \left| \Um f i {t \tau}{0} \right|^2
\end{eqnarray*}
In this expression, the (absolute) transition probabilities $\TP i f$ between two quantum states can be developed in series of the form:
\begin{eqnarray*}
\TP i f & = & \sum_{n=0}^\infty \TPn i f n \\
\TPn i f n & \eqdef & \sum_{n_1 + n_2 = n} \cc{\Umn f i {n_1} {t \tau}{0}} \Umn f i {n_2} {t \tau}{0}
\end{eqnarray*}
If $i$ and $f$ are plane wave states, these terms can be written using the scattering matrix as:
\begin{equation*}
\TPn i f n \eqdef \sum_{n_1 + n_2 = n} \cc{\Smn f i {n_1}} \Smn f i {n_2}
\end{equation*}

\section{Leading order transition probability}

The general form of the transition probability between plane wave modes of the field can be given without knowing much about the interaction term $\Hop'$. We assume here that the initial and final states of the field are plane waves of the form:
\begin{eqnarray*}
\ket{\Psi_i} & = & \ket{\bFQ {N_i} \pt \q \sp} \\
\ket{\Psi_f} & = & \ket{\bFQ {N_f} \pt \q \sp}
\end{eqnarray*}
The first interesting terms in the development of the transition probability are given in that case by:
\begin{eqnarray*}
\TPn i f 0 & \eqdef & \cc{\Smn f i 0} \Smn f i 0 = \delta_{f,i} \\
\TPn i f 1 & \eqdef & \cc{\Smn f i 0} \Smn f i 1 + \cc{\Smn f i 1} \Smn f i 0 = 0
\end{eqnarray*}
\begin{eqnarray*}
\TPn i f 2 & \eqdef & \cc{\Smn f i 0} \Smn f i 2 + \cc{\Smn f i 1} \Smn f i 1 + \cc{\Smn f i 2} \Smn f i 0 \\
& = & ( 2 \pi )^2 \frac{t - t_0} \h \left| H'_{f,i} \right|^2 \deltaE2{E_f - E_i} \\
&& -\delta_{f,i} ( 2 \pi )^2 \frac{t - t_0} \h \sum_{\bFQ {N_1} \pt \q \sp} \left| H'_{1,i} \right|^2 \deltaE2{E_1 - E_i}
\end{eqnarray*}
where the nascent delta function $\deltaE2{E}$ is defined as in appendix \ref{delta}.

\section{Higher order transition probability}

To the order $n \geq 2$, the transition probability between plane wave states $\ket{\bFQ {N_i} \pt \q \sp}$ and $\ket{\bFQ {N_f} \pt \q \sp}$ is given by:
\begin{eqnarray*}
\TPn i f n & \eqdef & \sum_{n_1 + n_2 = n} \cc{\Smn f i {n_1}} \Smn f i {n_2} \\
& = & \delta_{f,i} \sideset{}{_{k=1}^{n-1}}\sum_{\bFQ {N_k} \pt \q \sp} \left( \Smp n {i,\dotsc,i} + \cc{\Smp n {i,\dotsc,i}} \right) \\
&& + \sum_{\substack{n_1 + n_2 = n \\ n_1, n_2 \geq 1}} \cc{\Smn f i {n_1}} \Smn f i {n_2}
\end{eqnarray*}
The first term vanishes for $f \neq i$. The development of the last term involves a ``closed loop'' of length $n$ from $i$ to $i$ over $f$, i.e. a summation over $n - 2$ intermediate states $\ket{\bFQ {N_k} \pt \q \sp}$, where $k \in \IntRange{-n_1}{n_2}$, $\bFQ {N_0} \pt \q \sp = \bFQ {N_i} \pt \q \sp$ and $\bFQ {N_{-n_1}} \pt \q \sp = \bFQ {N_{n_2}} \pt \q \sp = \bFQ {N_f} \pt \q \sp$, and can be written as:
\begin{multline*}
\sum_{\substack{n_1 + n_2 = n \\ n_1, n_2 \geq 1}} \sideset{}{_{k=-n_1}^{n_2}}\sum_{\bFQ {N_k} \pt \q \sp} \left( \prod_{k = -n_1}^{n_2 - 1} H'_{k+1,k} \right) \cc{\SmE{n_1}{E_{-n_1}, \dotsc, E_0}} \SmE{n_2}{E_{n_2}, \dotsc, E_0}
\end{multline*}

To the third order, for instance, the transition probability for $f \neq i$ reads:
\begin{multline*}
\TPn i f 3 = ( 2 \pi )^3 \deltaE1{E_f - E_i} \\
\sum_{\bFQ {N_1} \pt \q \sp} \left[ \Im \left( H'_{i,f} H'_{f,1} H'_{1,i} \right ) \deltaE1{E_f - E_1} \deltaE1{E_1 - E_i} \right. \\
\left. + \Re \left( H'_{i,f} H'_{f,1} H'_{1,i} \right ) \frac{\deltaE1{E_f - E_i} - \cos{\pi \frac{t - t_0} \h ( E_f - E_1 )} \deltaE1{E_1 - E_i}}{\pi ( E_f - E_1 )} \right]
\end{multline*}
where the nascent delta function $\deltaE1{E}$ is defined as in appendix \ref{delta}.

\section{Ideal experimental setup}

We consider a scattering experiment designed to produce $n_f$ particles of type $\pt_j$, of wave vector $\q_j$ and of spin state $\sp_j$. To detect them all, a set of $n_f$ particle detectors $D_j$ is being used and we consider a single alternative: either all the detectors are activated or at least one of them isn't. The momentum of the detected particles is measured with an uncertainty given by the domain $\delta P_j$ of the momentum space in which particle $j$ could be found without changing the measurement result. The corresponding subset $\delta Q_j$ of values of $\q_j$ is given in the lattice reference frame by:
\begin{equation*}
\delta Q_j = \FLD \cap \frac \a \h \delta P_j
\end{equation*}
We assume that the corresponding subspace $\delta \F$ of $\H$ is given by:
\begin{eqnarray*}
\delta \F & = & \bigoplus_{(\delta \q_j)}^{\perp} \IC \dop{\Psi}_{f + (\delta \q_j)} \ket{O} \\
\dop{\Psi}_{f + (\delta \q_j)} & \eqdef & \prod_j \cop {\pt_j} {\q_j + \delta \q_j} {\sp_j}
\end{eqnarray*}
where the summation goes over all the $\delta \q_j$ verifying $\q_j + \delta \q_j \in \delta Q_j$ and where $\ket{O}$ describes the experimental setup, including measuring devices and the observer.

The probability that all the detectors are activated is then given by:
\begin{equation*}
\TP i {\delta \F} \eqdef \sum_{(\delta \q_j)} \TP i {f + (\delta \q_j)}
\end{equation*}
If the transition probability $\TP i {f + (\delta \q_j)}$, as a function of $(\delta \q_j)$, admits a continuation on $\IR^{3 n_f}$, an approximation of this sum can be obtained by taking the corresponding integral:
\begin{equation*}
\TP i {\delta \F} \approx \int_{\prod_j \delta P_j} \TP i {f + (\delta \q_j)} \left( (1 + 2 \N) \frac \a \h \right)^{3 n_f} \dtp_1 \dotsm \dtp_{n_f}
\end{equation*}
where $\delta \q_j \eqdef \frac \a \h \p_j - \q_j$ in the lattice reference frame.

In particular, if $\left| H'_{f + (\delta \q_j),i} \right|^2$ admits such a continuation and if $i$ and $f$ could be approximated by plane wave states, the leading order transition probability could be approximated, for $i \notin \delta \F$, by:
\begin{eqnarray*}
\TPn i {\delta \F} 2 & \approx & \int_{\prod_j \delta P_j} ( 2 \pi )^2 \frac{t - t_0} \h \left| H'_{f + (\delta \q_j),i} \right|^2 \deltaE2{E_{f + (\delta \q_j)} - E_i} \\
&& \left( (1 + 2 \N) \frac \a \h \right)^{3 n_f} \dtp_1 \dotsm \dtp_{n_f}
\end{eqnarray*}

\section{Quantum measurement}
\label{Quantum measurement}

Let us consider a simple thought experiment in order to illustrate how measurement processes take place: An excited atom decays by emitting a photon, which is detected by a photomultiplier read by an observer. In the initial state, at time $t_0$, the atom has just been switched to its excited state and hasn't decayed yet, the measurement apparatus is indicating that it hasn't yet detected any photon and the observer is waiting for the detector to get activated. We symbolize this situation by:
\begin{flushleft}
\ExcitedAtom\hspace*{3mm}\DetectorOff\hspace*{3mm}\ObserverOff
\end{flushleft}
The decay of the excited atom into a stable atom and a photon (via the QED processes described in chapter \ref{Quantum Electrodynamics}) first brings it into a quantum superposition of states, where both states coexist as a linear combination within the quantum state $\ket{\psi}$ of the universe. In both cases, the measurement apparatus is still inactive so far and the observer waiting. We symbolize this situation by:
\begin{flushleft}
\ExcitedAtom\hspace*{3mm}\DetectorOff\hspace*{3mm}\ObserverOff
\hspace*{3mm}\Superposition\hspace*{3mm}
\Atom\hspace*{3mm}\Photon\hspace*{3mm}\DetectorOff\hspace*{3mm}\ObserverOff
\end{flushleft}
Supposing, to simplify, that the photomultiplier has a detection efficiency of 100\%, it gets activated with certainty by the incoming photon after a certain delay $\Delta t$, coming thus itself into a quantum superposition of states. At the same time, the excited atom populates again the decayed atom state, as in the preceding step. In all three cases, the observer is still waiting that far. We symbolize this situation by:
\begin{flushleft}
\ExcitedAtom\hspace*{3mm}\DetectorOff\hspace*{3mm}\ObserverOff
\hspace*{3mm}\Superposition\hspace*{3mm}
\Atom\hspace*{3mm}\Photon\hspace*{3mm}\DetectorOff\hspace*{3mm}\ObserverOff
\hspace*{3mm}\Superposition\hspace*{3mm}
\Atom\hspace*{3mm}\DetectorOn\hspace*{3mm}\ObserverOff
\end{flushleft}
Finally, the observer is becoming aware of the fact that the detector has been activated and comes herself into a quantum superposition of states, so that four qualitatively different states coexist as a linear combination within the quantum state $\ket{\psi}$ of the universe. We symbolize this situation by:
\begin{flushleft}
\ExcitedAtom\hspace*{3mm}\DetectorOff\hspace*{3mm}\ObserverOff
\hspace*{3mm}\Superposition\hspace*{3mm}
\Atom\hspace*{3mm}\Photon\hspace*{3mm}\DetectorOff\hspace*{3mm}\ObserverOff
\hspace*{3mm}\Superposition\hspace*{3mm}
\Atom\hspace*{3mm}\DetectorOn\hspace*{3mm}\ObserverOff
\hspace*{3mm}\Superposition\hspace*{3mm}
\Atom\hspace*{3mm}\DetectorOn\hspace*{3mm}\ObserverOn
\end{flushleft}
This (purely material) situation lasts until the next process of collapse and collective mind selection takes place. The experienced consciousness state of the observer is then, randomly, either one of ``I am still waiting for the detector to get activated'' or ``I have seen the detector becoming activated''. In the first case, the quantum state of the universe becomes again:
\begin{flushleft}
\ExcitedAtom\hspace*{3mm}\DetectorOff\hspace*{3mm}\ObserverOff
\hspace*{3mm}\Superposition\hspace*{3mm}
\Atom\hspace*{3mm}\Photon\hspace*{3mm}\DetectorOff\hspace*{3mm}\ObserverOff
\hspace*{3mm}\Superposition\hspace*{3mm}
\Atom\hspace*{3mm}\DetectorOn\hspace*{3mm}\ObserverOff
\end{flushleft}
i.e. the detector has still been activated, but the observer didn't yet notice it. In the second case, the state of the universe becomes:
\begin{flushleft}
\Atom\hspace*{3mm}\DetectorOn\hspace*{3mm}\ObserverOn
\end{flushleft}
and the atom has definitely decayed. The probability $\delta P$ of this event to occur is very small, but the process of collapse and collective mind selection take place very often, with a period $\tau$, so that the decay of the atom should eventually be observed. The leading order approximation of this elementary probability takes the form:
\begin{equation*}
\delta P \approx ( 2 \pi )^2 \frac\tau\h \sum_f \left| H'_{f,i} \right|^2 \deltaE2{E_f - E_i}
\end{equation*}
where the Coulomb interaction term is supposed to have been shifted into the zeroth order Hamiltonian operator $\Hop_0$, where the summation runs over an orthonormal basis of eigenstates of this operator and where the delta function can be approximated by the (time independent) density of decay states (with regard to energy) around the ``allowed'' decay states conserving zeroth order energy. The duration until the decay is being observed follows a Poisson law and its mean value is given by the general formula:
\begin{equation*}
\left< t - t_0 \right> \approx \Delta t + \tau \frac{1 - \delta P}{\delta P}
\end{equation*}
It can be approximated in this case, since $\delta P \ll 1$, by:
\begin{equation*}
\left< t - t_0 \right> \approx \Delta t + \h \Big[ ( 2 \pi )^2 \sum_f \left| H'_{f,i} \right|^2 \deltaE2{E_f - E_i} \Big]^{-1}
\end{equation*}
Note that this result doesn't depend on the period $\tau$ of the collapse and collective mind selection process.
