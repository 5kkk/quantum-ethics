\chapter{Philosophical issues}
\label{Philosophical issues}

\section{The explanatory gap}

As it has been observed by science philosophers in the last century, the gap between our understanding level of physical-material and of mental phenomena has been continuously growing as the scientific community successfully focused on the development of the Relativity and Quantum theories. It is therefore quite understandable on a science psychological level that some cognitive neuroscientists may have been hoping to be able to explain one day all mental phenomena in terms of biophysical processes. However, even if we could describe one day the correspondence between mental states and quantum states of the brain, the question of knowing ``why'' some specific aspects of the material world correspond to certain mental experiences, and why some other aspects do not, would still remain open. This question is known as the ``explanatory gap'' and isn't to be answered by a theory focusing uniquely on the material world. The theory developed in this book addresses this issue in a threefold way. First, it gives a well-defined status to mental states, considered to be an aspect of reality on their own that isn't merely derived from the material one. This is expressed in the theory by the form  $\H \times \CMS$ of the set of the possible states of God. Second, it defines the form of the correspondence between material and mental states, which is given by the family of the supplementary subspaces $\H_\cms$ corresponding to each possible mental state $\cms$. This stresses the idea that individual mental experiences are not necessarily only related to material aspects of a single, individual brain, but that the collective mental experience globally relates to the quantum state of the world as a whole. Finally, the description of the random collapse process from a virtual into a real state of God gives a first explanation of what is happening at a material and at a mental level as a mental state is getting experienced.

\section{Skepticism}

According to philosophical skepticism, in the form of Descartes' \textit{Cogito Ergo Sum} argument in his \textit{Discourse on the Method}~\cite{Descartes1637} for instance, the one and only aspect of the world which we know beyond any doubt to be real is the experience of our present individual mind state, the `Cogito'. Nothing can guarantee us that the representations of the world carried by this mind state -- like our past experiences, the feeling of the permanence of our existence, the image of our body, of the outer world, of our relations to others -- have or have had any physical reality. In particular, it cannot be taken for granted that experimental evidence can be accumulated over the ages: Experimental science \textit{must} rely on the mere belief that the mental representations of what we consider to be accumulated experimental evidence are related to physical processes that really did happen in the past. Indeed, in physical terms, stating that I am experiencing some individual mind state $\ims_s$ only implies that the collective mind state $\cms = \bFM N \ims$ is such that $\bM N {\ims_s} \geq 1$ and that the quantum state $\ket \Psi$ of the universe belongs to $\H_{\ims_s}^+$. It doesn't necessarily imply that the past evolution of $\ket \Psi$ corresponds to the mental representation of past experiences in the mind state $\ims_s$. The physical theory presented in this book belongs therefore to the long tradition of philosophical skepticism insofar as it doubts the very possibility of experimental science.

%TODO: Philosophical issues: Personal identity
%\section{Personal identity}

%TODO: Philosophical issues: Free will
%\section{Free will}

%TODO: Philosophical issues: Qualia
%\section{Qualia}

\section{Materialism}

Materialism is the doctrine according to which the subjective experience of consciousness can be completely reduced to the corresponding physical-material processes happening within our brains and thus can be explained without involving any other level of reality than the purely material one. It is generally considered among philosophers as the daydream of a physicist absorbed by his study object and becoming blind for the reality of his own subjective experience. Nevertheless, it still has numerous supporters in today's scientific community. In the frame of the theory developed in this book, it could be formulated as the hypothesis that no individual mind state is possible, since this would be equivalent to denying the existence of the mental world, which is of another nature as the material one. Mathematically, this hypothesis can be expressed simply as $\H_{\cms_\vac} = \H$, so that no individual mind state is being experienced in any quantum state. Equivalently, this could be expressed in terms of collapse operators by $\op{\Pi}_{\cms_\vac} = \Id$, so that there is no collapse of the quantum state of the universe. Its evolution reduces therefore to its Hamiltonian part,
\begin{eqnarray*}
\ket{\Psi(t)} & = & \exp{- \i 2 \pi \frac{t - t_0} \h \Hop} \ket{\Psi(t_0)}
\end{eqnarray*}
and the stochastic process of collective mind selection do not apply.

Materialism in this context is facing the problem that it cannot satisfactorily explain how it is supposed to ``feel like'' in quantum states where brains happen to be in a quantum superposition of states corresponding to different states of consciousness. This would be the case for instance in a state of the form:
\begin{equation*}
\left(\sqrt{0.9}\ \dop{\Psi_\ims^\alpha} + \sqrt{0.1}\ \dop{\Psi_{\ims'}^{\alpha'}}\right) \H_{\cms_\vac}
\end{equation*}
where the brain states corresponding to the mental states $\bM 1 \ims$ and $\bM 1 {\ims'}$ are both present in a quantum superposition with the statistical weights 90\% and 10\%, respectively. There are two well-known ways of trying to escape this issue. In the no collapse theory of Everett, each consciousness state in the quantum superposition of a brain is supposed to be equally real as the others and to be experienced on its own. More precisely, these consciousness states are supposed to be statistically ``weighted'' in some (mysterious) way (since there isn't any random process taking place) by the square norm of the corresponding component of the quantum state of the universe, so that we are supposed to be more likely to experience one of them if it corresponds to a component with a greater square norm.

The second way of escaping the difficulties of materialism is to deny that there are ``noticeable'' quantum superpositions of consciousness states of the brain. This is basically the aim of all spontaneous collapse theories, which have been reviewed exhaustively by Angelo Bassi and GianCarlo Ghirardi in their report \textit{Dynamical Reduction Models}~\cite{Bassi2003}. Generally, the quantum state of the universe is supposed to collapse in such a way that the center of mass of macroscopic objects is practically always localized in a small region of space, so that we cannot notice its quantum fluctuations with our naked senses. As a consequence, insofar as our consciousness state is being mostly driven by sensory experience only, the states of consciousness corresponding to the components of a quantum superposition of brains are most likely to differ very little from another, so that it shouldn't really mind if we don't know exactly which one is being experienced.

\section{Solipsism}

The solipsist is convinced that she is (and must be) the only person in the universe who has a subjective mental experience. Solipsism makes thus unproblematic the fact that we are experiencing the mental world in the form of a single individual mind instead of experiencing the whole collective mind state directly. In the frame of the theory developed in this book, solipsism can be expressed as the hypothesis that the only possible collective mind states (apart from $\cms_\vac$) are of the form $\bM 1 \ims$, or in physical terms, that:
\begin{equation*}
\H = \H_{\cms_\vac} \operp \bigoplus_{\ims}^\perp \H_{\bM 1 \ims}
\end{equation*}
Of course, this hypothesis is logically perfectly correct, but it is utmost difficult to make it compatible with the idea that mind states are being realized physically by the presence of corresponding quantum states of brains. Even if one supposes that the solipsist's brain has something special that makes it differ from other brains that aren't being experienced as individual mind states, one faces the problem that a quantum state in which many ``copies'' of the solipsist's brain, corresponding to different individual mind states, would be present couldn't be related in a satisfactory way to a single individual mind state: It is unclear, for instance, if quantum states in a subspace of the form $\dop{\Psi_{\ims'}^{\alpha'}} \dop{\Psi_\ims^\alpha} \H_{\cms_\vac}$ should be experienced as the mental state $\bM 1 \ims$ or $\bM 1 {\ims'}$.

%TODO: Philosophical issues: Soul
%\section{Soul}

%TODO: Philosophical issues: God
%\section{God}

%TODO: Philosophical issues: Mental birth
%\section{Mental birth}
%Und was gab das den Frauen für eine wehmütige Schönheit, wenn sie schwanger waren und standen, und in ihrem großen Leib, auf welchem die schmalen Hände unwillkürlich liegen blieben, waren zwei Früchte: ein Kind und ein Tod. Kam das dichte, beinah nahrhafte Lächeln in ihrem ganz ausgeräumten Gesicht nicht davon her, daß sie manchmal meinten, es wüchsen beide?
%Rainer Maria Rilke: Die Aufzeichnungen des Malte Laurids Brigge

\section{Soul immortality theorem}

\paragraphtitle{Reminder} As stated in section \ref{Mental evolution}, Quantum Field Theory defines, for an arbitrary initial quantum state $\ket{\Psi_i} \neq 0$ realized at the initial time $t_i = 0$, the probability laws $\mathrm P_t(\cms_0, \dotsc, \cms_t; \ket{\Psi_i})$, where $t \in \IN$, that a given sequence of collective mind states $\cms_0, \dotsc, \cms_t$ is being experienced at times $0, \dotsc, t \tau$. These probability laws read:
\begin{equation*}
\mathrm P_t(\cms_0, \dotsc, \cms_t; \ket{\Psi_i}) = \bra{\Psi_i} \op \Pi_{\cms_0} \dop U_\tau \op \Pi_{\cms_1} \dotsm \dop U_\tau \op \Pi_{\cms_t} \op U_\tau \dotsm \op \Pi_{\cms_1} \op U_\tau \op \Pi_{\cms_0} \ket{\Psi_i} / \braket{\Psi_i}{\Psi_i}
\end{equation*}

\paragraphtitle{Definitions} An infinite sequence $(\cms_t)$, indexed on $t \in \IN$, is said to be ``dead'' if it is constant, \textit{i.e.} if $\cms_t = \cms_0$ for all $t \in \IN$; to be ``eventually dying'' if it becomes constant after a certain point, \textit{i.e.} if there exists a $t_1 \in \IN^*$ such that $\cms_t = \cms_{t_1}$ for all $t \geq t_1$, although $\cms_{t_0} \neq \cms_{t_1}$ for some $t_0 < t_1$; and to be ``immortal'' otherwise.

It is said to be ``certain'', for a given initial quantum state $\ket{\Psi_i} \neq 0$, if we have $\mathrm P_t(\cms_0, \dotsc, \cms_t; \ket{\Psi_i}) = 1$ for all $t \in \IN$; to be ``infinitely improbable'' if, for any initial quantum state $\ket{\Psi_i} \neq 0$, $\lim_{t \to \infty} \mathrm P_t(\cms_0, \dotsc, \cms_t; \ket{\Psi_i}) = 0$; and to be ``contingent'' otherwise.

A quantum state is said to be ``certainly dead'' if, taken as an initial state, one of the dead sequences is certain; to be ``possibly dead'' if a dead sequence is contingent; to be ``mortal'' if an eventually dying sequence is certain or at least contingent; and to be ``immortal'' otherwise.

\paragraphtitle{Lemma} Eventually dying sequences are infinitely improbable.

\paragraphtitle{Proof} An eventually dying sequence $(\cms_t)$ is characterized, per definition, by the existence of a $t_f \in \IN^*$ such that:
\begin{equation*}
t_f = \min \{ t \in \IN\ |\ \forall t' > t,\ \cms_{t'} = \cms_t \}
\end{equation*}
This sequence is, per definition, infinitely improbable if and only if, for any initial quantum state $\ket{\Psi_i} \neq 0$, we have:
\begin{equation*}
\mathrm P_t(\cms_0, \dotsc, \cms_t; \ket{\Psi_i}) \xrightarrow{t \to \infty} 0
\end{equation*}
The analysis of the asymptotic behavior can be simplified by using, for $t \geq t_f$, following factorization:
\begin{multline*}
\mathrm P_t(\cms_0, \dotsc, \cms_t; \ket{\Psi_i}) = \mathrm P_{t_f}(\cms_0, \dotsc, \cms_{t_f}; \ket{\Psi_i}) \\
\mathrm P_{t - t_f}(\cms_{t_f}, \dotsc, \cms_t; \op \Pi_{\cms_{t_f}} \op U_\tau \dotsm \op \Pi_{\cms_1} \op U_\tau \op \Pi_{\cms_0} \ket{\Psi_i})
\end{multline*}
The asymptotic behavior is governed by the second factor, which we will analyze under the generic form:
\begin{equation*}
\mathrm P_t(\cms, \dotsc, \cms; \ket{\Psi}) = \bra{\Psi} (\op \Pi_{\cms} \dop U_\tau)^t \op \Pi_{\cms} (\op U_\tau \op \Pi_{\cms})^t \ket{\Psi} / \braket{\Psi}{\Psi}
\end{equation*}
We will show that it admits a limit of the form:
\begin{equation*}
\mathrm P_t(\cms, \dotsc, \cms; \ket{\Psi}) \xrightarrow{t \to \infty} \bra{\Psi} \op \Pi_{\H_\cms^\infty} \ket{\Psi} / \braket{\Psi}{\Psi}
\end{equation*}
where $\H_\cms^\infty$ is a subspace of $\H_\cms$ that we will define subsequently. We have therefore:
\begin{multline*}
\mathrm P_t(\cms_0, \dotsc, \cms_t; \ket{\Psi_i}) \xrightarrow{t \to \infty} \\
\bra{\Psi_i} \op \Pi_{\cms_0} \dop U_\tau \op \Pi_{\cms_1} \dotsm \dop U_\tau \op \Pi_{\H_{\cms_{t_f}}^\infty} \op U_\tau \dotsm \op \Pi_{\cms_1} \op U_\tau \op \Pi_{\cms_0} \ket{\Psi_i} / \braket{\Psi_i}{\Psi_i}
\end{multline*}
and proving that $\op \Pi_{\H_\cms^\infty} \op U_\tau \op \Pi_{\cms'} = 0$ for any $\cms' \neq \cms$ will yield the conclusion.

\paragraphtitle{§} Let us first analyze the structure of a mental subspace $\H_\cms$. Let $\H_\cms^n$ be the subspace of the quantum states in $\H_\cms$ remaining in $\H_\cms$ after $1, \dotsc, n$ applications of the elementary evolution operator $\op U_\tau$. This operator being unitary, it is invertible, so that we can define these subspaces by:
\begin{equation*}
\H_\cms^n \eqdef \bigcap_{t = 0}^n \op U_\tau^{-t} \H_\cms
\end{equation*}
where $n \in \IN$. These subspaces are included in each other by construction, and we write $\H_\cms^\infty$ their intersection, given by:
\begin{equation*}
\H_\cms^\infty \eqdef \bigcap_{t = 0}^\infty \op U_\tau^{-t} \H_\cms
\end{equation*}

\paragraphtitle{§} $\H_\cms$ being finite dimensional, there must exist a finite number of distinct nested subspaces $\H_\cms^n$. Indeed, $\op U_\tau^{-1}$ being injective, we have
\begin{equation*}
\H_\cms^{n+1} = \H_\cms \cap \op U_\tau^{-1} \H_\cms^n
\end{equation*}
for any $n \in \IN$; if there exists a $n \in \IN$ such that $\H_\cms^n = \H_\cms^{n+1}$, we have therefore by recurrence $\H_\cms^n = \H_\cms^{n'} = \H_\cms^\infty$ for any $n' \geq n$. If there didn't exist such a $n \in \IN$, the inclusions $\H_\cms^{n+1} \subset \H_\cms^n$ would all be strict, so that the dimension of $\H_\cms^0 = \H_\cms$ would be infinite, which isn't the case. There must also exist a $n_\cms \in \IN$ such that
\begin{equation*}
n_\cms = \min \{ n \in \IN\ |\ \H_\cms^n = \H_\cms^{n+1} \}
\end{equation*}
and we have $\H_\cms^{n_\cms} = \H_\cms^\infty$.

\paragraphtitle{§} In the special case $n_\cms = 0$, it is trivial to show that $\mathrm P_t(\cms, \dotsc, \cms; \ket{\Psi})$ tends to $\bra{\Psi} \op \Pi_{\H_\cms^\infty} / \braket{\Psi}{\Psi}$ for any initial quantum state $\ket{\Psi} \in \H \setminus \{ 0 \}$. In this case, we have $\H_\cms^\infty = \H_\cms$ and we will see that  $\mathrm P_t(\cms, \dotsc, \cms; \ket{\Psi}) = \bra{\Psi} \op \Pi_{\H_\cms^\infty} \ket{\Psi} / \braket{\Psi}{\Psi}$ for any $t \in \IN$. The assertion holds for $t = 0$ per definition, and $\op \Pi_\cms \ket{\Psi} \in \H_\cms$ since $\op \Pi_\cms$ is a projection operator on $\H_\cms$. For any quantum state $\ket{\Psi_\cms} \in \H_\cms$, $\op U_\tau \ket{\Psi_\cms}$ has the same norm as $\ket{\Psi_\cms}$ because of the unitarity of $\op U_\tau$, and since $\H_\cms = \H_\cms^\infty \subset \op U_\tau^{-1} \H_\cms$, $\op U_\tau \ket{\Psi_\cms} \in \H_\cms$, so that $\op \Pi_\cms \op U_\tau \ket{\Psi_\cms}$ is equal to $\op U_\tau \ket{\Psi_\cms}$ and has the same norm as $\ket{\Psi_\cms}$, too. It is easy then to prove by recurrence that $(\op \Pi_\cms \op U_\tau)^t \ket{\Psi_\cms}$ is equal to $\op U_\tau^t \ket{\Psi_\cms}$ and has therefore the same norm as $\ket{\Psi_\cms}$ for any $t \in \IN$, so that we have:
\begin{eqnarray*}
\mathrm P_t(\cms, \dotsc, \cms; \ket{\Psi}) & = & \bra{\Psi} \op \Pi_\cms (\dop U_\tau \op \Pi_\cms)^t (\op \Pi_\cms \op U_\tau)^t \op \Pi_\cms \ket{\Psi} / \braket{\Psi}{\Psi} \\
& = & \bra{\Psi} \op \Pi_\cms \ket{\Psi} / \braket{\Psi}{\Psi} = \bra{\Psi} \op \Pi_{\H_\cms^\infty} \ket{\Psi} / \braket{\Psi}{\Psi}
\end{eqnarray*}
for any quantum state $\ket{\Psi} \in \H \setminus \{ 0 \}$ and any $t \in \IN$.

\paragraphtitle{§} We assume from now on $n_\cms \geq 1$, so that the supplementary subspace $\H_\cms^{\infty \perp}$ of $\H_\cms^{\infty}$ in $\H_\cms$, defined by:
\begin{equation*}
\H_\cms = \H_\cms^{\infty} \operp \H_\cms^{\infty \perp}
\end{equation*}
isn't reduced to the zero subspace. We will consider the endomorphism $\op U_\cms$ induced by $\op U_\tau$ on $\H_\cms$, given by:
\begin{equation*}
\op U_\cms \eqdef \op \Pi_\cms \op U_\tau \op \Pi_\cms
\end{equation*}
With this operator, we can write for any $t \in \IN^*$ and any quantum state $\ket{\Psi} \in \H \setminus \{ 0 \}$:
\begin{equation*}
\mathrm P_t(\cms, \dotsc, \cms; \ket{\Psi}) = \bra{\Psi} \dop U_\cms^t \op U_\cms^t \ket{\Psi} / \braket{\Psi}{\Psi}
\end{equation*}
and we will see that the subspaces $\H_\cms^n$ can be characterized, for any $n \in \IN^*$, by:
\begin{equation*}
\H_\cms^n = \{ \ket{\Psi} \in \H\ |\ \bra{\Psi} \dop U_\cms^n \op U_\cms^n \ket{\Psi} = \braket{\Psi}{\Psi} \}
\end{equation*}

\paragraphtitle{§} Let us prove it by recurrence. $\op \Pi_\cms$ being the orthogonal projection operator on $\H_\cms$, for any quantum state $\ket{\Psi} \in \H$, $\op \Pi_\cms \ket{\Psi}$ and $\ket{\Psi}$ have the same norm if and only if $\ket{\Psi} \in \H_\cms$, otherwise $\bra{\Psi} \op \Pi_\cms \ket{\Psi} < \braket{\Psi}{\Psi}$. $\op U_\tau$ being unitary, it preserves the norm of $\op \Pi_\cms \ket{\Psi}$. Consequently, $\op U_\cms \ket{\Psi}$ and $\ket{\Psi}$ have the same norm if and only if $\ket{\Psi} \in \H_\cms$ and $\op U_\tau \op \Pi_\cms \ket{\Psi} \in \H_\cms$, otherwise $\bra{\Psi} \dop U_\cms \op U_\cms \ket{\Psi} < \braket{\Psi}{\Psi}$. For any $\ket{\Psi} \in \H_\cms$, $\op \Pi_\cms \ket{\Psi} = \ket{\Psi}$, so that this condition is equivalent to $\ket{\Psi} \in \H_\cms \cap \op U_\tau^{-1} \H_\cms = \H_\cms^1$. We have therefore:
\begin{equation*}
\H_\cms^1 = \{ \ket{\Psi} \in \H\ |\ \bra{\Psi} \dop U_\cms \op U_\cms \ket{\Psi} = \braket{\Psi}{\Psi} \}
\end{equation*}
which proves the assertion for $n = 1$. Let us assume that, for a given $n \in \IN^*$, $\op U_\cms^n \ket{\Psi}$ and $\ket{\Psi}$ have the same norm if and only if $\ket{\Psi} \in \H_\cms^n$, whereas $\bra{\Psi} \dop U_\cms^n \op U_\cms^n \ket{\Psi} < \braket{\Psi}{\Psi}$ otherwise. It can then be proved like above that $\op U_\cms^{n+1} \ket{\Psi}$ and $\ket{\Psi}$ have the same norm if and only if $\ket{\Psi} \in \H_\cms^n$, $\op U_\cms^n \ket{\Psi} \in \H_\cms$ and $\op U_\tau \op U_\cms^n \ket{\Psi} \in \H_\cms$, whereas $\bra{\Psi} \dop U_\cms^{n+1} \op U_\cms^{n+1} \ket{\Psi} < \braket{\Psi}{\Psi}$ otherwise. For any $\ket{\Psi} \in \H_\cms^n$, $\op U_\cms^n \ket{\Psi} = \op U_\tau^n \ket{\Psi}$, so that this condition is equivalent to $\ket{\Psi} \in \H_\cms^n \cap \op U_\tau^{-n} \H_\cms \cap \op U_\tau^{-(n+1)} \H_\cms = \H_\cms^{n+1}$, which proves the assertion for any $n \in \IN^*$ by recurrence.

\paragraphtitle{§} As a consequence, since $\H_\cms^{n_\cms} = \H_\cms^\infty$ and $\H_\cms^{\infty \perp}$ are disjoint  by construction, we have in particular:
\begin{equation*}
\forall \ket{\Psi} \in \H_\cms^{\infty \perp} \setminus \{ 0 \},\ \mathrm P_{n_\cms}(\cms, \dotsc, \cms; \ket{\Psi}) < 1
\end{equation*}
The function $\mathrm P_{n_\cms}(\cms, \dotsc, \cms; \ket{\Psi})$ is continuous on $\H \setminus \{ 0 \}$ and, being constant on the rays $\IC^* \ket{\Psi}$, its maximum on $\H_\cms^{\infty \perp} \setminus \{ 0 \}$ can be evaluated on the unit sphere of this subspace. Because of the finite dimensionality of $\H_\cms^{\infty \perp}$, the unit sphere is compact, so that this maximum exists and is being reached. There exists therefore a $p_\cms \in [0, 1[$ such that:
\begin{equation*}
p_\cms^{n_\cms} = \max \{ \mathrm P_{n_\cms}(\cms, \dotsc, \cms; \ket{\Psi})\ | \ket{\Psi} \in \H_\cms^{\infty \perp} \setminus \{ 0 \} \}
\end{equation*}
This will allow us to set an upper bound to $\mathrm P_t(\cms, \dotsc, \cms; \ket{\Psi})$ for any $t \geq n_\cms$. Per factorization, for any quantum state $\ket{\Psi} \in \H \setminus \{ 0 \}$ and any $k \in \IN^*$, we have:
\begin{equation*}
\mathrm P_{k n_\cms}(\cms, \dotsc, \cms; \ket{\Psi}) = \mathrm P_{n_\cms}(\cms, \dotsc, \cms; \ket{\Psi})^k
\end{equation*}
and, $\mathrm P_t(\cms, \dotsc, \cms; \ket{\Psi})$ being a decreasing function of $t$, we have more generally for any $t \geq n_\cms$:
\begin{equation*}
\mathrm P_t(\cms, \dotsc, \cms; \ket{\Psi}) \leq \mathrm P_{n_\cms}(\cms, \dotsc, \cms; \ket{\Psi})^{\lfloor t / n_\cms \rfloor}
\end{equation*}
where $\lfloor \cdot \rfloor$ denotes the floor function. In particular, for any quantum state $\ket{\Psi} \in \H_\cms^{\infty \perp} \setminus \{ 0 \}$, we have the estimation:
\begin{equation*}
\mathrm P_t(\cms, \dotsc, \cms; \ket{\Psi}) \leq p_\cms^{n_\cms \lfloor t / n_\cms \rfloor} \xrightarrow{t \to \infty} 0
\end{equation*}

\paragraphtitle{§} It is easy to show that, for any quantum state $\ket{\Psi} \in \H$ orthogonal to $\H_\cms^\infty$ and for any mental state $\cms' \in \CMS$, $\op U_{\cms'} \ket{\Psi}$ is still orthogonal to $\H_\cms^\infty$. For any mental state $\cms' \neq \cms$, this is trivial because $\op U_{\cms'} \ket{\Psi} \in \H_{\cms'}$, which is orthogonal to the mental subspace $\H_\cms$ and \textit{a fortiori} to its subspace $\H_\cms^\infty$. We assume from now on $\cms' = \cms$. The decomposition $\op \Pi_\cms = \op \Pi_{\H_\cms^\infty} + \op \Pi_{\H_\cms^{\infty \perp}}$ reduces to $\op \Pi_\cms \ket{\Psi} = \op \Pi_{\H_\cms^{\infty \perp}} \ket{\Psi} \in \H_\cms^{\infty \perp}$ since $\ket{\Psi}$ is orthogonal to $\H_\cms^\infty$. $\op U_\tau$ being unitary, it preserves orthogonality, so that $\op U_\tau \op \Pi_\cms \ket{\Psi}$ is orthogonal to $\op U_\tau \H_\cms^\infty = \H_\cms^\infty$. The above decomposition of $\op \Pi_\cms$ reduces therefore again to $\op \Pi_\cms \op U_\tau \op \Pi_\cms \ket{\Psi} = \op \Pi_{\H_\cms^{\infty \perp}} \op U_\tau \op \Pi_\cms \ket{\Psi}$, which proves that $\op U_\cms \ket{\Psi} \in \H_\cms^{\infty \perp}$.

\paragraphtitle{§} For any quantum state $\ket{\Psi} \in \H$, any mental state $\cms \in \CMS$ and any $n \in \IN^*$, $\op U_\cms^n \ket{\Psi}$ can thus be decomposed by linearity into the orthogonal sum of a quantum state $\op U_\tau^n \op \Pi_{\H_\cms^\infty} \ket{\Psi} \in \H_\cms^\infty$ and of a quantum state $\op U_\cms^n \op \Pi_{\H_\cms^{\infty \perp}} \ket{\Psi} \in \H_\cms^{\infty \perp}$. Hence its squared norm can be decomposed into:
\begin{equation*}
\bra{\Psi} \dop U_\cms^n \op U_\cms^n \ket{\Psi} = \bra{\Psi} \op \Pi_{\H_\cms^\infty} \ket{\Psi} + \bra{\Psi} \op \Pi_{\H_\cms^{\infty \perp}} \dop U_\cms^n \op U_\cms^n \op \Pi_{\H_\cms^{\infty \perp}} \ket{\Psi}
\end{equation*}
If the quantum state $\ket{\Psi} \in \H \setminus \{ 0 \}$ is orthogonal to $\H_\cms^{\infty \perp}$, the second term is zero, so that we have for any $t \in \IN^*$:
\begin{equation*}
\mathrm P_t(\cms, \dotsc, \cms; \ket{\Psi}) = \bra{\Psi} \op \Pi_{\H_\cms^\infty} \ket{\Psi} / \braket{\Psi}{\Psi}
\end{equation*}
Otherwise, for any quantum state $\ket{\Psi}$ having a non-zero component in $\H_\cms^{\infty \perp}$, we can write for any $t \in \IN$:
\begin{multline*}
\mathrm P_t(\cms, \dotsc, \cms; \ket{\Psi}) = \bra{\Psi} \op \Pi_{\H_\cms^\infty} \ket{\Psi} / \braket{\Psi}{\Psi} \\
+ \mathrm P_t(\cms, \dotsc, \cms; \op \Pi_{\H_\cms^{\infty \perp}} \ket{\Psi}) \bra{\Psi} \op \Pi_{\H_\cms^{\infty \perp}} \ket{\Psi} / \braket{\Psi}{\Psi}
\end{multline*}
so that we have more generally:
\begin{equation*}
\forall \ket{\Psi} \in \H \setminus \{ 0 \},\ \mathrm P_t(\cms, \dotsc, \cms; \ket{\Psi}) \xrightarrow{t \to \infty} \bra{\Psi} \op \Pi_{\H_\cms^\infty} \ket{\Psi} / \braket{\Psi}{\Psi}
\end{equation*}

\paragraphtitle{§} It is now easy to conclude. For any sequence $(\cms_t)$ eventually dying at $t_f$, as stated above, we have for any initial quantum state $\ket{\Psi_i} \in \H \setminus \{ 0 \}$:
\begin{multline*}
\mathrm P_t(\cms_0, \dotsc, \cms_t; \ket{\Psi_i}) \xrightarrow{t \to \infty} \\
\bra{\Psi_i} \op \Pi_{\cms_0} \dop U_\tau \dotsm \op \Pi_{\cms_{t_f-1}} \dop U_\tau \op \Pi_{\H_{\cms_{t_f}}^\infty} \op U_\tau \op \Pi_{\cms_{t_f-1}} \dotsm \op U_\tau \op \Pi_{\cms_0} \ket{\Psi_i} / \braket{\Psi_i}{\Psi_i}
\end{multline*}
Now we've already seen that, for any quantum state $\ket{\Psi}$ orthogonal to $\H_{\cms_{t_f}}^\infty$, $\op U_\tau \ket{\Psi}$ is orthogonal to this subspace too. Since $\cms_{t_f-1} \neq \cms_{t_f}$ by hypothesis, the subspaces $\H_\cms_{t_f-1}$ and $\H_{\cms_{t_f}}^\infty \subset \H_\cms_{t_f}$ are orthogonal to each other, so that we have $\op \Pi_{\H_{\cms_{t_f}}^\infty} \op U_\tau \op \Pi_{\cms_{t_f-1}} = 0$. As a consequence,
\begin{equation*}
\forall \ket{\Psi_i} \in \H \setminus \{ 0 \},\ \mathrm P_t(\cms_0, \dotsc, \cms_t; \ket{\Psi_i}) \xrightarrow{t \to \infty} 0
\end{equation*}
Any eventually dying sequence is therefore infinitely improbable, \textit{Q.~E.~D.}

\paragraphtitle{Theorem} No quantum state is mortal.

\paragraphtitle{Proof} This is an immediate consequence of the lemma.

\paragraphtitle{Corollary} There will be with certainty extra-terrestrial forms of conscious life.

\paragraphtitle{Proof} The mental state $\cms_{t_1}$ we are experiencing right now at time $t_1$ is presumably not the only mental state we have ever experienced, so we take for granted that there has already been a time $t_0 < t_1$ in the past where a mental state $\cms_{t_0} \neq \cms_{t_1}$ has been experienced. The probability that we keep experiencing the same mental state $\cms_{t_1}$ forever from now on can be expessed as a conditional probability of the form:
\begin{equation*}
\lim_{t \to \infty} \mathrm P_t(\cms_0, \dotsc, \cms_t; \ket{\Psi_i}) / \mathrm P_{t_1}(\cms_0, \dotsc, \cms_{t_1}; \ket{\Psi_i})
\end{equation*}
where the sequence $(\cms_t)$ is eventually dying. As a consequence of the lemma, this probability is zero, hence we will experience with certainty a different mental state in the future. This theorem holds for any time $t > t_1$, too: There will be with certainty a time $t' > t$ such that $\cms_{t'} \neq \cms_t$. So basically, once the experienced mental state has changed for the first time in the history of the universe, it will ``keep moving'' forever, although it is still possible that there are long periods of ``mental inactivity'' in between.

This applies in particular to the moment when the Sun, in about five billions of years, will have eventually evolved to a red giant and destroyed the Earth together with every terrestrial form of conscious life. By then, if there isn't yet any extra-terrestrial form of conscious life, the experienced mental state should be $\cms_\vac$, \textit{i.e.} the absence of any mental experience. But this state cannot last forever, as we have just seen. If we call ``extra-terrestrial form of conscious life'' the material substrate of the mental state that would get experienced next, then there will be some with certainty.

%TODO: Philosophical issues: Thought experiments
%\chapter{Thought experiments}

%TODO: Philosophical issues: Double-slit experiment
%\section{Double-slit experiment}

%TODO: Philosophical issues: Schrödinger's cat
%\section{Schrödinger's cat}

%TODO: Philosophical issues: Wigner's friend
%\section{Wigner's friend}

%TODO: Philosophical issues: EPR paradox
%\section{EPR paradox}
