\chapter{Philosophical issues}
\label{Philosophical issues}

\section{The explanatory gap}

As it has been observed by science philosophers in the last century, the gap between our understanding level of physical-material and of mental phenomena has been continuously growing as the scientific community successfully focused on the development of the Relativity and Quantum theories.
It is therefore quite understandable on a science psychological level that some cognitive neuroscientists may have been hoping to be able to explain one day all mental phenomena in terms of biophysical processes.
However, even if we could describe one day the correspondence between mental states and quantum states of the brain, the question of knowing ``why'' some specific aspects of the material world correspond to certain mental experiences, and why some other aspects do not, would still remain open.
This question is known as the ``explanatory gap'' and isn't to be answered by a theory focusing uniquely on the material world.
The theory developed in this book addresses this issue in a threefold way.
First, it gives a well-defined status to mental states, considered to be an aspect of reality on their own that isn't merely derived from the material one.
This is expressed in the theory by the form  $\H \times \CMS$ of the set of the possible states of God.
Second, it defines the form of the correspondence between material and mental states, which is given by the family of the supplementary subspaces $\H_\cms$ corresponding to each possible mental state $\cms$.
This stresses the idea that individual subjective experiences are not necessarily only related to material aspects of a single, individual brain, but that the totality of all subjective experiences globally relates to the quantum state of the universe.
Finally, the description of the random collapse process from a virtual into a real state of God gives a first explanation of what is happening at a material and at a mental level as a mental state is getting experienced.

\section{Skepticism}

According to philosophical skepticism, in the form of Descartes' \textit{Cogito Ergo Sum} argument in his \textit{Discourse on the Method}~\cite{Descartes1637} for instance, the one and only aspect of the world which we know beyond any doubt to be real is our present subjective experience, the `Cogito'.
Nothing can guarantee us that the representations of the world it carries -- like our past experiences, the feeling of the permanence of our existence, the image of our body, of the outer world, of our relations to others -- have or have had any physical reality.
In particular, it cannot be taken for granted that experimental evidence can be accumulated over the ages: Experimental science \textit{must} rely on the mere belief that the mental representations of what we consider to be accumulated experimental evidence are related to physical processes that really did happen in the past.
Indeed, in physical terms, stating that I am having some subjective experience $\ims_s$ only implies that the mental state $\cms = \bFM N \ims$ is such that $\bM N {\ims_s} \geq 1$ and that the quantum state $\ket \Psi$ of the universe belongs to $\H_{\ims_s}^+$.
It doesn't necessarily imply that the past evolution of $\ket \Psi$ corresponds to the mental representation of past events in the subjective experience $\ims_s$.
The physical theory presented in this book belongs therefore to the long tradition of philosophical skepticism insofar as it doubts the very possibility of experimental science.

%TODO: Philosophical issues: Personal identity
%\section{Personal identity}

%TODO: Philosophical issues: Free will
%\section{Free will}

%TODO: Philosophical issues: Qualia
%\section{Qualia}

\section{Materialism}

Materialism is the doctrine according to which the subjective experience of consciousness can be completely reduced to the corresponding physical-material processes happening within our brains and thus can be explained without involving any other level of reality than the purely material one.
It is generally considered among philosophers as the daydream of a physicist absorbed by his study object and becoming blind for the reality of his own subjective experience.
Nevertheless, it still has numerous supporters in today's scientific community.
In the frame of the theory developed in this book, it could be formulated as the hypothesis that no subjective experience is possible, since this would be equivalent to denying the existence of the mental world, which is of another nature as the material one.
Mathematically, this hypothesis can be expressed simply as $\H_{\cms_\vac} = \H$, so that no subject is having any subjective experience in any quantum state.
Equivalently, this could be expressed in terms of collapse operators by $\op{\Pi}_{\cms_\vac} = \Id$, so that there is no collapse of the quantum state of the universe.
Its evolution reduces therefore to its Hamiltonian part,
\begin{eqnarray*}
\ket{\Psi(t)} & = & \exp{- \i 2 \pi \frac{t - t_0} \h \Hop} \ket{\Psi(t_0)}
\end{eqnarray*}
and the stochastic process of mental state selection do not apply.

Materialism in this context is facing the problem that it cannot satisfactorily explain how it is supposed to ``feel like'' in quantum states where brains happen to be in a quantum superposition of states corresponding to different states of consciousness.
This would be the case for instance in a state of the form:
\begin{equation*}
\left(\sqrt{0.9}\ \dop{\Psi_\ims^\alpha} + \sqrt{0.1}\ \dop{\Psi_{\ims'}^{\alpha'}}\right) \H_{\cms_\vac}
\end{equation*}
where the brain states corresponding to the mental states $\bM 1 \ims$ and $\bM 1 {\ims'}$ are both present in a quantum superposition with the statistical weights 90\% and 10\%, respectively.
There are two well-known ways of trying to escape this issue.
In the no collapse theory of Everett, each consciousness state in the quantum superposition of a brain is supposed to be equally real as the others and to be experienced on its own.
More precisely, these consciousness states are supposed to be statistically ``weighted'' in some (mysterious) way (since there isn't any random process taking place) by the square norm of the corresponding component of the quantum state of the universe, so that we are supposed to be more likely to experience one of them if it corresponds to a component with a greater square norm.

The second way of escaping the difficulties of materialism is to deny that there are ``noticeable'' quantum superpositions of consciousness states of the brain.
This is basically the aim of all spontaneous collapse theories, which have been reviewed exhaustively by Angelo Bassi and GianCarlo Ghirardi in their report \textit{Dynamical Reduction Models}~\cite{Bassi2003}.
Generally, the quantum state of the universe is supposed to collapse in such a way that the center of mass of macroscopic objects is practically always localized in a small region of space, so that we cannot notice its quantum fluctuations with our naked senses.
As a consequence, insofar as our consciousness state is being mostly driven by sensory experience only, the states of consciousness corresponding to the components of a quantum superposition of brains are most likely to differ very little from another, so that it shouldn't really mind if we don't know exactly which one is being experienced.

\section{Solipsism}

The solipsist is convinced that she is (and must be) the only person in the universe who has a subjective experience.
Solipsism makes thus unproblematic the fact that we are experiencing the mental world in the form of a single subjective experience instead of experiencing the whole mental state directly.
In the frame of the theory developed in this book, solipsism can be expressed as the hypothesis that the only possible mental states (apart from $\cms_\vac$) are of the form $\bM 1 \ims$, or in physical terms, that:
\begin{equation*}
\H = \H_{\cms_\vac} \operp \bigoplus_{\ims}^\perp \H_{\bM 1 \ims}
\end{equation*}
Of course, this hypothesis is logically perfectly correct, but it is utmost difficult to make it compatible with the idea that mental states are being realized physically by the presence of corresponding quantum states of brains.
Even if one supposes that the solipsist's brain has something special that makes it differ from other brains that aren't giving rise to a subjective experience, one faces the problem that a quantum state in which many ``copies'' of the solipsist's brain, corresponding to different subjective experiences, would be present couldn't be related in a satisfactory way to a single subjective experience: It is unclear, for instance, if quantum states in a subspace of the form $\dop{\Psi_{\ims'}^{\alpha'}} \dop{\Psi_\ims^\alpha} \H_{\cms_\vac}$ should be experienced as $\bM 1 \ims$ or $\bM 1 {\ims'}$.

%TODO: Philosophical issues: Soul
%\section{Soul}

%TODO: Philosophical issues: God
%\section{God}

%TODO: Philosophical issues: Mental birth
%\section{Mental birth}
%Und was gab das den Frauen für eine wehmütige Schönheit, wenn sie schwanger waren und standen, und in ihrem großen Leib, auf welchem die schmalen Hände unwillkürlich liegen blieben, waren zwei Früchte: ein Kind und ein Tod. Kam das dichte, beinah nahrhafte Lächeln in ihrem ganz ausgeräumten Gesicht nicht davon her, daß sie manchmal meinten, es wüchsen beide?
%Rainer Maria Rilke: Die Aufzeichnungen des Malte Laurids Brigge

\section{Soul immortality theorem}

\paragraphtitle{Reminder} As stated in section \ref{Mental evolution}, Quantum Field Theory defines, for an arbitrary initial quantum state $\ket{\Psi_i} \neq 0$ realized at the initial time $t_i = 0$, the probability laws $\mathrm P_t(\cms_0, \dotsc, \cms_t; \ket{\Psi_i})$, where $t \in \IN$, that a given sequence of mental states $\cms_0, \dotsc, \cms_t$ is being experienced at times $0, \dotsc, t \tau$.
These probability laws read:
\begin{equation*}
\mathrm P_t(\cms_0, \dotsc, \cms_t; \ket{\Psi_i}) = \bra{\Psi_i} \op \Pi_{\cms_0} \dop U_\tau \op \Pi_{\cms_1} \dotsm \dop U_\tau \op \Pi_{\cms_t} \op U_\tau \dotsm \op \Pi_{\cms_1} \op U_\tau \op \Pi_{\cms_0} \ket{\Psi_i} / \braket{\Psi_i}{\Psi_i}
\end{equation*}

\paragraphtitle{Definitions} An infinite sequence $(\cms_t)$, indexed on $t \in \IN$, is said to be ``dead'' if it is constant, \textit{i.e.} if $\cms_t = \cms_0$ for all $t \in \IN$; to be ``eventually dying'' if it becomes constant after a certain point, \textit{i.e.} if there exists a $t_1 \in \IN^*$ such that $\cms_t = \cms_{t_1}$ for all $t \geq t_1$, although $\cms_{t_0} \neq \cms_{t_1}$ for some $t_0 < t_1$; and to be ``immortal'' otherwise.

It is said to be ``certain'', for a given initial quantum state $\ket{\Psi_i} \neq 0$, if we have $\mathrm P_t(\cms_0, \dotsc, \cms_t; \ket{\Psi_i}) = 1$ for all $t \in \IN$; to be ``infinitely improbable'' if $\lim_{t \to \infty} \mathrm P_t(\cms_0, \dotsc, \cms_t; \ket{\Psi_i}) = 0$; and to be ``contingent'' otherwise.

A quantum state is said to be ``certainly dead'' if, taken as an initial state, one of the dead sequences is certain; to be ``possibly dead'' if a dead sequence is contingent; to be ``mortal'' if an eventually dying sequence is certain or at least contingent; and to be ``immortal'' otherwise.

\paragraphtitle{Lemma} Eventually dying sequences are infinitely improbable.

\paragraphtitle{Proof} An eventually dying sequence $(\cms_t)$ is characterized, per definition, by the existence of a $t_f \in \IN^*$ such that:
\begin{equation*}
t_f = \min \{ t \in \IN\ |\ \forall t' > t,\ \cms_{t'} = \cms_t \}
\end{equation*}
This sequence is, per definition, infinitely improbable if and only if, for any initial quantum state $\ket{\Psi_i} \neq 0$, we have:
\begin{equation*}
\mathrm P_t(\cms_0, \dotsc, \cms_t; \ket{\Psi_i}) \xrightarrow{t \to \infty} 0
\end{equation*}
The analysis of the asymptotic behavior can be simplified by using, for $t \geq t_f$, following factorization:
\begin{multline*}
\mathrm P_t(\cms_0, \dotsc, \cms_t; \ket{\Psi_i}) = \mathrm P_{t_f}(\cms_0, \dotsc, \cms_{t_f}; \ket{\Psi_i}) \\
\mathrm P_{t - t_f}(\cms_{t_f}, \dotsc, \cms_t; \op \Pi_{\cms_{t_f}} \op U_\tau \dotsm \op \Pi_{\cms_1} \op U_\tau \op \Pi_{\cms_0} \ket{\Psi_i})
\end{multline*}
The asymptotic behavior is governed by the second factor, which we will analyze under the generic form:
\begin{equation*}
\mathrm P_t(\cms, \dotsc, \cms; \ket{\Psi}) = \bra{\Psi} (\op \Pi_{\cms} \dop U_\tau)^t \op \Pi_{\cms} (\op U_\tau \op \Pi_{\cms})^t \ket{\Psi} / \braket{\Psi}{\Psi}
\end{equation*}
We will show that it admits a limit of the form:
\begin{equation*}
\mathrm P_t(\cms, \dotsc, \cms; \ket{\Psi}) \xrightarrow{t \to \infty} \bra{\Psi} \op \Pi_{\H_\cms^\infty} \ket{\Psi} / \braket{\Psi}{\Psi}
\end{equation*}
where $\H_\cms^\infty$ is a subspace of $\H_\cms$ that we will define subsequently.
We have therefore:
\begin{multline*}
\mathrm P_t(\cms_0, \dotsc, \cms_t; \ket{\Psi_i}) \xrightarrow{t \to \infty} \\
\bra{\Psi_i} \op \Pi_{\cms_0} \dop U_\tau \op \Pi_{\cms_1} \dotsm \dop U_\tau \op \Pi_{\H_{\cms_{t_f}}^\infty} \op U_\tau \dotsm \op \Pi_{\cms_1} \op U_\tau \op \Pi_{\cms_0} \ket{\Psi_i} / \braket{\Psi_i}{\Psi_i}
\end{multline*}
and proving that $\op \Pi_{\H_\cms^\infty} \op U_\tau \op \Pi_{\cms'} = 0$ for any $\cms' \neq \cms$ will yield the conclusion.

\paragraphtitle{§} Let us first analyze the structure of a mental subspace $\H_\cms$.
Let $\H_\cms^n$ be the subspace of the quantum states in $\H_\cms$ remaining in $\H_\cms$ after $1, \dotsc, n$ applications of the elementary evolution operator $\op U_\tau$.
This operator being unitary, it is invertible, so that we can define these subspaces by:
\begin{equation*}
\H_\cms^n \eqdef \bigcap_{t = 0}^n \op U_\tau^{-t} \H_\cms
\end{equation*}
where $n \in \IN$.
These subspaces are included in each other by construction, and we write $\H_\cms^\infty$ their intersection, given by:
\begin{equation*}
\H_\cms^\infty \eqdef \bigcap_{t = 0}^\infty \op U_\tau^{-t} \H_\cms
\end{equation*}

\paragraphtitle{§} $\H_\cms$ being finite dimensional, there must exist a finite number of distinct nested subspaces $\H_\cms^n$.
Indeed, $\op U_\tau^{-1}$ being injective, we have
\begin{equation*}
\H_\cms^{n+1} = \H_\cms \cap \op U_\tau^{-1} \H_\cms^n
\end{equation*}
for any $n \in \IN$; if there exists a $n \in \IN$ such that $\H_\cms^n = \H_\cms^{n+1}$, we have therefore by recurrence $\H_\cms^n = \H_\cms^{n'} = \H_\cms^\infty$ for any $n' \geq n$.
If there didn't exist such a $n \in \IN$, the inclusions $\H_\cms^{n+1} \subset \H_\cms^n$ would all be strict, so that the dimension of $\H_\cms^0 = \H_\cms$ would be infinite, which isn't the case.
There must also exist a $n_\cms \in \IN$ such that
\begin{equation*}
n_\cms = \min \{ n \in \IN\ |\ \H_\cms^n = \H_\cms^{n+1} \}
\end{equation*}
and we have $\H_\cms^{n_\cms} = \H_\cms^\infty$.

\paragraphtitle{§} In the special case $n_\cms = 0$, it is trivial to show that $\mathrm P_t(\cms, \dotsc, \cms; \ket{\Psi})$ tends to $\bra{\Psi} \op \Pi_{\H_\cms^\infty} \ket{\Psi} / \braket{\Psi}{\Psi}$ for any initial quantum state $\ket{\Psi} \neq 0$.
In this case, we have $\H_\cms^\infty = \H_\cms$ and we will see that  $\mathrm P_t(\cms, \dotsc, \cms; \ket{\Psi}) = \bra{\Psi} \op \Pi_{\H_\cms^\infty} \ket{\Psi} / \braket{\Psi}{\Psi}$ for any $t \in \IN$.
The assertion holds for $t = 0$ per definition, and $\op \Pi_\cms \ket{\Psi} \in \H_\cms$ since $\op \Pi_\cms$ is a projection operator on $\H_\cms$.
For any quantum state $\ket{\Psi_\cms} \in \H_\cms$, $\op U_\tau \ket{\Psi_\cms}$ has the same norm as $\ket{\Psi_\cms}$ because of the unitarity of $\op U_\tau$, and since $\H_\cms = \H_\cms^\infty \subset \op U_\tau^{-1} \H_\cms$, $\op U_\tau \ket{\Psi_\cms}$ belongs to $\H_\cms$, so that $\op \Pi_\cms \op U_\tau \ket{\Psi_\cms}$ is equal to $\op U_\tau \ket{\Psi_\cms}$ and has the same norm as $\ket{\Psi_\cms}$, too.
It is easy then to prove by recurrence that $(\op \Pi_\cms \op U_\tau)^t \ket{\Psi_\cms}$ is equal to $\op U_\tau^t \ket{\Psi_\cms}$ and has therefore the same norm as $\ket{\Psi_\cms}$ for any $t \in \IN$, so that we have:
\begin{eqnarray*}
\mathrm P_t(\cms, \dotsc, \cms; \ket{\Psi}) & = & \bra{\Psi} \op \Pi_\cms (\dop U_\tau \op \Pi_\cms)^t (\op \Pi_\cms \op U_\tau)^t \op \Pi_\cms \ket{\Psi} / \braket{\Psi}{\Psi} \\
& = & \bra{\Psi} \op \Pi_\cms \ket{\Psi} / \braket{\Psi}{\Psi} = \bra{\Psi} \op \Pi_{\H_\cms^\infty} \ket{\Psi} / \braket{\Psi}{\Psi}
\end{eqnarray*}
for any quantum state $\ket{\Psi} \neq 0$ and any $t \in \IN$.

\paragraphtitle{§} We assume from now on $n_\cms \geq 1$, so that the supplementary subspace $\H_\cms^{\infty \perp}$ of $\H_\cms^{\infty}$ in $\H_\cms$, defined by:
\begin{equation*}
\H_\cms = \H_\cms^{\infty} \operp \H_\cms^{\infty \perp}
\end{equation*}
isn't reduced to the zero subspace.
We will consider the endomorphism $\op U_\cms$ induced by $\op U_\tau$ on $\H_\cms$, given by:
\begin{equation*}
\op U_\cms \eqdef \op \Pi_\cms \op U_\tau \op \Pi_\cms
\end{equation*}
With this operator, we can write for any $t \in \IN^*$ and any quantum state $\ket{\Psi} \neq 0$:
\begin{equation*}
\mathrm P_t(\cms, \dotsc, \cms; \ket{\Psi}) = \bra{\Psi} \dop U_\cms^t \op U_\cms^t \ket{\Psi} / \braket{\Psi}{\Psi}
\end{equation*}
and we will see that the subspaces $\H_\cms^n$ can be characterized, for any $n \in \IN^*$, by:
\begin{equation*}
\H_\cms^n = \{ \ket{\Psi} \in \H\ |\ \bra{\Psi} \dop U_\cms^n \op U_\cms^n \ket{\Psi} = \braket{\Psi}{\Psi} \}
\end{equation*}

\paragraphtitle{§} Let us prove it by recurrence.
$\op \Pi_\cms$ being the orthogonal projection operator on $\H_\cms$, for any quantum state $\ket{\Psi} \in \H$, $\op \Pi_\cms \ket{\Psi}$ and $\ket{\Psi}$ have the same norm if and only if $\ket{\Psi} \in \H_\cms$, otherwise $\bra{\Psi} \op \Pi_\cms \ket{\Psi} < \braket{\Psi}{\Psi}$.
$\op U_\tau$ being unitary, it preserves the norm of $\op \Pi_\cms \ket{\Psi}$.
Consequently, $\op U_\cms \ket{\Psi}$ and $\ket{\Psi}$ have the same norm if and only if $\ket{\Psi} \in \H_\cms$ and $\op U_\tau \op \Pi_\cms \ket{\Psi} \in \H_\cms$, otherwise $\bra{\Psi} \dop U_\cms \op U_\cms \ket{\Psi} < \braket{\Psi}{\Psi}$.
For any $\ket{\Psi} \in \H_\cms$, $\op \Pi_\cms \ket{\Psi} = \ket{\Psi}$, so that this condition is equivalent to $\ket{\Psi} \in \H_\cms \cap \op U_\tau^{-1} \H_\cms = \H_\cms^1$.
We have therefore:
\begin{equation*}
\H_\cms^1 = \{ \ket{\Psi} \in \H\ |\ \bra{\Psi} \dop U_\cms \op U_\cms \ket{\Psi} = \braket{\Psi}{\Psi} \}
\end{equation*}
which proves the assertion for $n = 1$.
Let us assume that, for a given $n \in \IN^*$, $\op U_\cms^n \ket{\Psi}$ and $\ket{\Psi}$ have the same norm if and only if $\ket{\Psi} \in \H_\cms^n$, whereas $\bra{\Psi} \dop U_\cms^n \op U_\cms^n \ket{\Psi} < \braket{\Psi}{\Psi}$ otherwise.
It can then be proved like above that $\op U_\cms^{n+1} \ket{\Psi}$ and $\ket{\Psi}$ have the same norm if and only if $\ket{\Psi} \in \H_\cms^n$, $\op U_\cms^n \ket{\Psi} \in \H_\cms$ and $\op U_\tau \op U_\cms^n \ket{\Psi} \in \H_\cms$, whereas $\bra{\Psi} \dop U_\cms^{n+1} \op U_\cms^{n+1} \ket{\Psi} < \braket{\Psi}{\Psi}$ otherwise.
For any $\ket{\Psi} \in \H_\cms^n$, $\op U_\cms^n \ket{\Psi} = \op U_\tau^n \ket{\Psi}$, so that this condition is equivalent to $\ket{\Psi} \in \H_\cms^n \cap \op U_\tau^{-n} \H_\cms \cap \op U_\tau^{-(n+1)} \H_\cms = \H_\cms^{n+1}$, which proves the assertion for any $n \in \IN^*$ by recurrence.

\paragraphtitle{§} As a consequence, since $\H_\cms^{n_\cms} = \H_\cms^\infty$ and $\H_\cms^{\infty \perp}$ are disjoint  by construction, we have in particular:
\begin{equation*}
\forall \ket{\Psi} \in \H_\cms^{\infty \perp} \setminus \{ 0 \},\ \mathrm P_{n_\cms}(\cms, \dotsc, \cms; \ket{\Psi}) < 1
\end{equation*}
The function $\mathrm P_{n_\cms}(\cms, \dotsc, \cms; \ket{\Psi})$ is continuous on $\H \setminus \{ 0 \}$ and, being constant on the rays $\IC^* \ket{\Psi}$, its maximum on $\H_\cms^{\infty \perp} \setminus \{ 0 \}$ can be evaluated on the unit sphere of this subspace.
Because of the finite dimensionality of $\H_\cms^{\infty \perp}$, the unit sphere is compact, so that this maximum exists and is being reached.
There exists therefore a $p_\cms \in [0, 1[$ such that:
\begin{equation*}
p_\cms^{n_\cms} = \max \{ \mathrm P_{n_\cms}(\cms, \dotsc, \cms; \ket{\Psi})\ | \ket{\Psi} \in \H_\cms^{\infty \perp} \setminus \{ 0 \} \}
\end{equation*}
This will allow us to set an upper bound to $\mathrm P_t(\cms, \dotsc, \cms; \ket{\Psi})$ for any $t \geq n_\cms$.
Per factorization, for any quantum state $\ket{\Psi} \in \H \setminus \{ 0 \}$ and any $k \in \IN^*$, we have:
\begin{equation*}
\mathrm P_{k n_\cms}(\cms, \dotsc, \cms; \ket{\Psi}) = \mathrm P_{n_\cms}(\cms, \dotsc, \cms; \ket{\Psi})^k
\end{equation*}
and, $\mathrm P_t(\cms, \dotsc, \cms; \ket{\Psi})$ being a decreasing function of $t$, we have more generally for any $t \geq n_\cms$:
\begin{equation*}
\mathrm P_t(\cms, \dotsc, \cms; \ket{\Psi}) \leq \mathrm P_{n_\cms}(\cms, \dotsc, \cms; \ket{\Psi})^{\lfloor t / n_\cms \rfloor}
\end{equation*}
where $\lfloor \cdot \rfloor$ denotes the floor function.
In particular, for any quantum state $\ket{\Psi} \in \H_\cms^{\infty \perp} \setminus \{ 0 \}$, we have the estimation:
\begin{equation*}
\mathrm P_t(\cms, \dotsc, \cms; \ket{\Psi}) \leq p_\cms^{n_\cms \lfloor t / n_\cms \rfloor} \xrightarrow{t \to \infty} 0
\end{equation*}

\paragraphtitle{§} It is easy to show that, for any quantum state $\ket{\Psi} \in \H$ orthogonal to $\H_\cms^\infty$ and for any mental state $\cms' \in \CMS$, $\op U_{\cms'} \ket{\Psi}$ is still orthogonal to $\H_\cms^\infty$.
For any mental state $\cms' \neq \cms$, this is trivial because $\op U_{\cms'} \ket{\Psi} \in \H_{\cms'}$, which is orthogonal to the mental subspace $\H_\cms$ and \textit{a fortiori} to its subspace $\H_\cms^\infty$.
We assume from now on $\cms' = \cms$.
The decomposition $\op \Pi_\cms = \op \Pi_{\H_\cms^\infty} + \op \Pi_{\H_\cms^{\infty \perp}}$ reduces to $\op \Pi_\cms \ket{\Psi} = \op \Pi_{\H_\cms^{\infty \perp}} \ket{\Psi} \in \H_\cms^{\infty \perp}$ since $\ket{\Psi}$ is orthogonal to $\H_\cms^\infty$.
$\op U_\tau$ being unitary, it preserves orthogonality, so that $\op U_\tau \op \Pi_\cms \ket{\Psi}$ is orthogonal to $\op U_\tau \H_\cms^\infty = \H_\cms^\infty$.
The above decomposition of $\op \Pi_\cms$ reduces therefore again to $\op \Pi_\cms \op U_\tau \op \Pi_\cms \ket{\Psi} = \op \Pi_{\H_\cms^{\infty \perp}} \op U_\tau \op \Pi_\cms \ket{\Psi}$, which proves that $\op U_\cms \ket{\Psi} \in \H_\cms^{\infty \perp}$.

\paragraphtitle{§} For any quantum state $\ket{\Psi} \in \H$, any mental state $\cms \in \CMS$ and any $n \in \IN^*$, $\op U_\cms^n \ket{\Psi}$ can thus be decomposed by linearity into the orthogonal sum of a quantum state $\op U_\tau^n \op \Pi_{\H_\cms^\infty} \ket{\Psi} \in \H_\cms^\infty$ and of a quantum state $\op U_\cms^n \op \Pi_{\H_\cms^{\infty \perp}} \ket{\Psi} \in \H_\cms^{\infty \perp}$.
Hence its squared norm can be decomposed into:
\begin{equation*}
\bra{\Psi} \dop U_\cms^n \op U_\cms^n \ket{\Psi} = \bra{\Psi} \op \Pi_{\H_\cms^\infty} \ket{\Psi} + \bra{\Psi} \op \Pi_{\H_\cms^{\infty \perp}} \dop U_\cms^n \op U_\cms^n \op \Pi_{\H_\cms^{\infty \perp}} \ket{\Psi}
\end{equation*}
If the quantum state $\ket{\Psi} \in \H \setminus \{ 0 \}$ is orthogonal to $\H_\cms^{\infty \perp}$, the second term is zero, so that we have for any $t \in \IN^*$:
\begin{equation*}
\mathrm P_t(\cms, \dotsc, \cms; \ket{\Psi}) = \bra{\Psi} \op \Pi_{\H_\cms^\infty} \ket{\Psi} / \braket{\Psi}{\Psi}
\end{equation*}
Otherwise, for any quantum state $\ket{\Psi}$ having a non-zero component in $\H_\cms^{\infty \perp}$, we can write for any $t \in \IN$:
\begin{multline*}
\mathrm P_t(\cms, \dotsc, \cms; \ket{\Psi}) = \bra{\Psi} \op \Pi_{\H_\cms^\infty} \ket{\Psi} / \braket{\Psi}{\Psi} \\
+ \mathrm P_t(\cms, \dotsc, \cms; \op \Pi_{\H_\cms^{\infty \perp}} \ket{\Psi}) \bra{\Psi} \op \Pi_{\H_\cms^{\infty \perp}} \ket{\Psi} / \braket{\Psi}{\Psi}
\end{multline*}
so that we have more generally:
\begin{equation*}
\forall \ket{\Psi} \in \H \setminus \{ 0 \},\ \mathrm P_t(\cms, \dotsc, \cms; \ket{\Psi}) \xrightarrow{t \to \infty} \bra{\Psi} \op \Pi_{\H_\cms^\infty} \ket{\Psi} / \braket{\Psi}{\Psi}
\end{equation*}

\paragraphtitle{§} It is now easy to conclude.
For any sequence $(\cms_t)$ eventually dying at $t_f$, as stated above, we have for any initial quantum state $\ket{\Psi_i} \in \H \setminus \{ 0 \}$:
\begin{multline*}
\mathrm P_t(\cms_0, \dotsc, \cms_t; \ket{\Psi_i}) \xrightarrow{t \to \infty} \\
\bra{\Psi_i} \op \Pi_{\cms_0} \dop U_\tau \dotsm \op \Pi_{\cms_{t_f-1}} \dop U_\tau \op \Pi_{\H_{\cms_{t_f}}^\infty} \op U_\tau \op \Pi_{\cms_{t_f-1}} \dotsm \op U_\tau \op \Pi_{\cms_0} \ket{\Psi_i} / \braket{\Psi_i}{\Psi_i}
\end{multline*}
Now we've already seen that, for any quantum state $\ket{\Psi}$ orthogonal to $\H_{\cms_{t_f}}^\infty$, $\op U_\tau \ket{\Psi}$ is orthogonal to this subspace too.
Since $\cms_{t_f-1} \neq \cms_{t_f}$ by hypothesis, the subspaces $\H_\cms_{t_f-1}$ and $\H_{\cms_{t_f}}^\infty \subset \H_\cms_{t_f}$ are orthogonal to each other, so that we have $\op \Pi_{\H_{\cms_{t_f}}^\infty} \op U_\tau \op \Pi_{\cms_{t_f-1}} = 0$.
As a consequence,
\begin{equation*}
\forall \ket{\Psi_i} \in \H \setminus \{ 0 \},\ \mathrm P_t(\cms_0, \dotsc, \cms_t; \ket{\Psi_i}) \xrightarrow{t \to \infty} 0
\end{equation*}
Any eventually dying sequence is therefore infinitely improbable, \textit{Q.~E.~D.}

\paragraphtitle{Remark} This result is actually a quite intuitive consequence of the finite dimensionality of the Hilbert space $\H$.
Precisely, the key argument -- the compactness of the unit sphere of $\H_\cms^{\infty \perp}$ -- requires that these subspaces -- the ``accessible'' part of the mental subspaces -- be finite dimensional.
This condition being given, the result would still hold if the ``inaccessible'' part $\H_\cms^\infty$ of the mental subspaces were infinite dimensional, and/or if there were an infinite number of possible mental states.

\paragraphtitle{Theorem} No quantum state is mortal.

\paragraphtitle{Proof} This is an immediate consequence of the lemma.

\paragraphtitle{Corollary} There will be with certainty extra-terrestrial forms of conscious life.

\paragraphtitle{Proof} The mental state $\cms_{t_1}$ we are experiencing right now at time $t_1$ is presumably not the only mental state we have ever experienced, so we take for granted that there has already been a time $t_0 < t_1$ in the past where a mental state $\cms_{t_0} \neq \cms_{t_1}$ has been experienced.
The probability that we keep experiencing the same mental state $\cms_{t_1}$ forever from now on can be expressed as a conditional probability of the form:
\begin{equation*}
\lim_{t \to \infty} \mathrm P_t(\cms_0, \dotsc, \cms_t; \ket{\Psi_i}) / \mathrm P_{t_1}(\cms_0, \dotsc, \cms_{t_1}; \ket{\Psi_i})
\end{equation*}
where the sequence $(\cms_t)$ is eventually dying.
As a consequence of the lemma, this probability is zero, hence we will experience with certainty a different mental state in the future.
This theorem holds for any time $t > t_1$, too: There will be with certainty a time $t' > t$ such that $\cms_{t'} \neq \cms_t$.
So basically, once the experienced mental state has changed for the first time in the history of the universe, it will ``keep moving'' forever, although it is still possible that there are long periods of ``mental inactivity'' in between.

This applies in particular to the moment when the Sun, in about five billions of years, will have eventually evolved to a red giant and made the Earth an unsuitable place for any form of conscious life.
By then, if there isn't yet any extra-terrestrial form of conscious life, the experienced mental state should be $\cms_\vac$, \textit{i.e.} the absence of any mental experience.
But this state cannot last forever, as we have just seen.
If we call ``extra-terrestrial form of conscious life'' the material substrate of the mental state that would get experienced next, then there will be some with certainty.

\paragraphtitle{Commentaries} Proving such theorems in a book on the foundations of physics could be taken as a provocation, but actually I included them here because they are obvious consequences of the formalism and are independent of the details of the physical interactions.
They show clearly enough, I think, both the philosophical potential and the dangers of any well-defined interpretation of Quantum Physics.
These theorems are in no way a proof of the reality of life after death or of the existence of UFOs, but they would have good chances to get over-interpreted if they would happen to be vulgarized.
Developing well-defined interpretations of Quantum Physics obviously has the potential of addressing philosophical issues which are of interest to the public, and it would be a pity not to investigate them from a naturalistic point of view.

\section{Reincarnation theorem}

Generalizing the soul immortality theorem, we will derive here a general theorem on the recurrence of mental states, that we will call ``reincarnation theorem'' because of its similarity with usual conceptions of reincarnation.
A discussion of its precise meaning follows the proof.

\paragraphtitle{Notations} Let $\CMS_0$ be a non empty subset of $\CMS$ and let us write $\H_{\CMS_0}$ the corresponding mental subspace, given by:
\begin{equation*}
\H_{\CMS_0} \eqdef \bigoplus_{\cms \in \CMS_0}^\perp \H_\cms
\end{equation*}
Let us define by recurrence, for any $n \in \IN$, the operators $\op P_{\CMS_0}^{(n)}$ by:
\begin{eqnarray*}
\op P_{\CMS_0}^{(0)} & \eqdef & \op \Pi_{\CMS_0} \eqdef \sum_{\cms \in \CMS_0} \op \Pi_\cms \\
\op P_{\CMS_0}^{(n+1)} & \eqdef & \sum_{\cms \in \CMS_0} \op \Pi_\cms \dop U_\tau \op P_{\CMS_0}^{(n)} \op U_\tau \op \Pi_\cms
\end{eqnarray*}
Finally, let us define by recurrence, for any $n \in \IN$, the subspaces $\H_{\CMS_0}^n$ by:
\begin{equation*}
\H_{\CMS_0}^0 \eqdef \H_{\CMS_0}
\end{equation*}
\begin{equation*}
\H_{\CMS_0}^{n+1} \eqdef \H_{\CMS_0}^n \cap \{ \ket \Psi \in \H\ |\ \forall \cms_0, \dotsc, \cms_n \in \CMS_0,\ \op U_\tau \op \Pi_{\cms_n} \dotsm \op U_\tau \op \Pi_{\cms_0} \ket \Psi \in \H_{\CMS_0} \}
\end{equation*}
Let us write $\H_{\CMS_0}^\infty$ their intersection, $\H_{\CMS_0}^{\infty \perp}$ its supplementary subspace in $\H_{\CMS_0}$, and $\op \Pi_{\H_{\CMS_0}^\infty}$ the orthogonal projection operator on $\H_{\CMS_0}^\infty$.

\paragraphtitle{Remarks} For any initial quantum state $\ket \Psi \neq 0$, $\bra \Psi \op P_{\CMS_0}^{(n)} \ket \Psi / \braket \Psi \Psi$ represents the probability that the experienced mental state belongs to $\CMS_0$ at all times $t = 0, \dotsc, n$.
The operators $\op P_{\CMS_0}^{(n)}$ are hermitian and therefore diagonalizable on an orthogonal basis with real eigenvalues; these eigenvalues being probabilities, they lie between 0 and 1.
The eigenspace for the eigenvalue 1 can be easily identified with $\H_{\CMS_0}^n$.
The subspace $\H_{\CMS_0}^\infty$ is therefore the set of all initial quantum states for which the experienced mental state belongs with certainty to $\CMS_0$ at all times.

\paragraphtitle{First lemma} For any quantum state $\ket \Psi \in \H$ orthogonal to $\H_{\CMS_0}^\infty$, and for all mental states $\cms \in \CMS$, the quantum states $\op U_\tau \ket \Psi$ and $\op \Pi_\cms \ket \Psi$ are orthogonal to $\H_{\CMS_0}^\infty$, too.

\paragraphtitle{Second lemma} The sequence of operators $(\op P_{\CMS_0}^{(n)})$ converges towards $\op \Pi_{\H_{\CMS_0}^\infty}$.

\paragraphtitle{Theorem} An initial mental state $\cms_i \in \CMS$ being given, for any initial quantum state $\ket \Psi \in \H_{\cms_i} \setminus \{0\}$, the probability that the mental state $\cms_i$ be experienced again in the future equals 1.

\paragraphtitle{Proof of the first lemma} Let us prove first that the subspace $\H_{\CMS_0}^\infty$ is identical to its image $\op U_\tau \H_{\CMS_0}^\infty$.
It is easy to see that, for any $n \in \IN$, the subspace $\op U_\tau \H_{\CMS_0}^{1+n}$ is included in $\H_{\CMS_0}^n$.
Indeed, any quantum state $\ket \Psi \in \H_{\CMS_0}^{1+n}$ belongs per definition to $\H_{\CMS_0}$ and verifies, for all mental states $\cms_0, \dotsc, \cms_n \in \CMS_0$ and for any $t \in \IntRange{1}{n}$, $\op U_\tau \op \Pi_{\cms_t} \dotsm \op U_\tau \op \Pi_{\cms_0} \ket \Psi \in \H_{\CMS_0}$.
Summing up over all $\cms_0 \in \CMS_0$ yields $\op U_\tau \op \Pi_{\cms_t} \dotsm \op U_\tau \op \Pi_{\CMS_0} \ket \Psi \in \H_{\CMS_0}$ and, since $\ket \Psi \in \H_{\CMS_0}$, $\op U_\tau \op \Pi_{\cms_t} \dotsm \op U_\tau \ket \Psi \in \H_{\CMS_0}$.
The quantum state $\op U_\tau \ket \Psi$ belongs also to $\H_{\CMS_0}^n$.
We have therefore $\op U_\tau \H_{\CMS_0}^{1+n} \subset \H_{\CMS_0}^n$.
Now $\H_{\CMS_0}^\infty$ is the intersection of all subspaces $\H_{\CMS_0}^n$ for $n \in \IN$, so its image $\op U_\tau \H_{\CMS_0}^\infty$ is included in the intersection of their images $\op U_\tau \H_{\CMS_0}^n$, and a fortiori in the intersection of the images $\op U_\tau \H_{\CMS_0}^{1+n}$, where $n \in \IN$.
Since they are themselves included in $\H_{\CMS_0}^n$, we have $\op U_\tau \H_{\CMS_0}^\infty \subset \H_{\CMS_0}^\infty$.
The subspace $\H_{\CMS_0}^\infty$ being finite dimensional and the endomorphism $\op U_\tau$ injective, we have also $\dim \op U_\tau \H_{\CMS_0}^\infty = \dim \H_{\CMS_0}^\infty$.
We have therefore $\op U_\tau \H_{\CMS_0}^\infty = \H_{\CMS_0}^\infty$.

\paragraphtitle{§} We can now prove the first part of the lemma: For any quantum state $\ket \Psi \in \H$ orthogonal to $\H_{\CMS_0}^\infty$, the quantum state $\op U_\tau \ket \Psi$ is orthogonal to $\H_{\CMS_0}^\infty$, too.
Indeed, the elementary evolution operator $\op U_\tau$ being unitary, the quantum state $\op U_\tau \ket \Psi$ is orthogonal to $\op U_\tau \H_{\CMS_0}^\infty$, which is identical to $\H_{\CMS_0}^\infty$, as we have just seen.

\paragraphtitle{§} Let us consider now, for any $n \in \IN$,  a quantum state $\ket \Psi \in \H_{\CMS_0}^n$ and a mental state $\cms \in \CMS$.
We will show by recurrence on $n$ that $\op \Pi_\cms \ket \Psi$ belongs to $\H_{\CMS_0}^n$, too.
If $\cms \in \CMS_0$, it is clear that $\op \Pi_\cms \ket \Psi$ belongs to $\H_{\CMS_0}^0$, because $\op \Pi_\cms$ is a projection operator on $\H_\cms$, which is a subspace of $\H_{\CMS_0} = \H_{\CMS_0}^0$.
If $\cms \notin \CMS_0$, $\op \Pi_\cms \ket \Psi$ is zero since $\ket \Psi$ belongs to $\H_{\CMS_0}$ by hypothesis, which is orthogonal to $\H_\cms$.
So in all cases, $\op \Pi_\cms \ket \Psi \in \H_{\CMS_0}^0$.
Let us assume that the assertion holds for a given $n \in \IN$.
To prove it at the next rang $n + 1$, it is sufficient to show that, for any mental states $\cms_0, \dotsc, \cms_n \in \CMS_0$, the quantum state $\op U_\tau \op \Pi_{\cms_n} \dotsm \op U_\tau \op \Pi_{\cms_0} \op \Pi_\cms \ket \Psi$ belongs to $\H_{\CMS_0}$.
If $\cms_0 \neq \cms$, this quantum state is zero since $\op \Pi_{\cms_0}$ and $\op \Pi_\cms$ are orthogonal projection operators on two orthogonal subspaces $\H_{\cms_0}$ and $\H_\cms$.
If $\cms_0 = \cms$, considering that, $\op \Pi_{\cms_0}$ being a projection operator, $\op \Pi_{\cms_0}^2 = \op \Pi_{\cms_0}$, this quantum state belongs to $\H_{\CMS_0}$ by hypothesis, which proves the assertion for all $n \in \IN$ by recurrence.
We have therefore $\op \Pi_\cms \H_{\CMS_0}^n \subset \H_{\CMS_0}^n$ for any mental state $\cms \in \CMS$ and all $n \in \IN$.

\paragraphtitle{§} One can see easily then that, for any mental state $\cms \in \CMS$, the projection $\op \Pi_\cms \H_{\CMS_0}^\infty$ of the subspace $\H_{\CMS_0}^\infty$ is included in $\H_{\CMS_0}^\infty$.
Indeed, since $\H_{\CMS_0}^\infty$ is the intersection of the subspaces $\H_{\CMS_0}^n$, its image $\op \Pi_\cms \H_{\CMS_0}^\infty$ is included in the intersection of the images $\op \Pi_\cms \H_{\CMS_0}^n$, which are themselves included in the subspaces $\H_{\CMS_0}^n$, as we have just seen.
Their intersection is therefore included in the intersection of the subspaces $\H_{\CMS_0}^n$, which is $\H_{\CMS_0}^\infty$.
We have therefore $\op \Pi_\cms \H_{\CMS_0}^\infty \subset \H_{\CMS_0}^\infty$ for any mental state $\cms \in \CMS$.

\paragraphtitle{§} This will allow us to prove that the subspace $\H_{\CMS_0}^\infty$ can be decomposed as the orthogonal sum of its projections $\op \Pi_\cms \H_{\CMS_0}^\infty$ for $\cms \in \CMS_0$.
This sum is orthogonal since the projections are included in the orthogonal subspaces $\H_\cms$, respectively.
It is obvious that $\H_{\CMS_0}^\infty$ is included in the sum, since any quantum state $\ket \Psi \in \H_{\CMS_0}^\infty$ belongs in particular to $\H_{\CMS_0}$, so that $\ket \Psi = \op \Pi_{\CMS_0} \ket \Psi = \sum_{\cms \in \CMS_0} \op \Pi_\cms \ket \Psi$.
The inverse inclusion also holds since, as we have just seen, the projections are subspaces of $\H_{\CMS_0}^\infty$.
We have therefore $\H_{\CMS_0}^\infty = \operp_{\cms \in \CMS_0} \op \Pi_\cms \H_{\CMS_0}^\infty$.

\paragraphtitle{§} We can conclude from this result that the subspace $\H_{\CMS_0}^{\infty \perp}$ can be decomposed too as the orthogonal sum of its projections $\op \Pi_\cms \H_{\CMS_0}^{\infty \perp}$ for $\cms \in \CMS_0$.
Indeed, the subspace $\H_{\CMS_0}$ being the orthogonal sum of the mental subspaces $\H_\cms$ for $\cms \in \CMS_0$, and the orthogonal components $\op \Pi_\cms \H_{\CMS_0}^\infty$ of $\H_{\CMS_0}^\infty$ being respectively included in the mental subspaces $\H_\cms$, the supplementary subspace of $\H_{\CMS_0}^\infty$ in $\H_{\CMS_0}$ can be decomposed as the orthogonal sum of the respective supplementary subspaces of $\op \Pi_\cms \H_{\CMS_0}^\infty$ in $\H_\cms$ -- which are therefore its respective projections on the mental subspaces $\H_\cms$.
We have therefore $\H_{\CMS_0}^{\infty \perp} = \operp_{\cms \in \CMS_0} \op \Pi_\cms \H_{\CMS_0}^{\infty \perp}$.

\paragraphtitle{§} We can now prove the second part of the lemma: For any quantum state $\ket \Psi \in \H$ orthogonal to $\H_{\CMS_0}^\infty$, and any mental state $\cms \in \CMS$, the quantum state $\op \Pi_\cms \ket \Psi$ is orthogonal to $\H_{\CMS_0}^\infty$, too.
Indeed, the supplementary subspace of $\H_{\CMS_0}^\infty$ is the orthogonal sum of its supplementary subspace in $\H_{\CMS_0}$, which is $\H_{\CMS_0}^{\infty \perp} = \operp_{\cms \in \CMS_0} \op \Pi_\cms \H_{\CMS_0}^{\infty \perp}$, and of the supplementary subspace of $\H_{\CMS_0}$ itself, which is $\operp_{\cms \in \CMS \setminus \CMS_0} \H_\cms$.
So if $\cms \notin \CMS_0$, the quantum state $\op \Pi_\cms \ket \Psi$, which belongs to $\H_\cms$, is orthogonal to $\H_{\CMS_0}$, and if $\cms \in \CMS_0$, the quantum state $\op \Pi_\cms \ket \Psi$ belongs to $\op \Pi_\cms \H_{\CMS_0}^{\infty \perp}$, which is a subspace of $\H_{\CMS_0}^{\infty \perp}$ and is therefore orthogonal to $\H_{\CMS_0}^\infty$.
For any mental state $\cms \in \CMS$, the quantum state $\op \Pi_\cms \ket \Psi$ is therefore orthogonal to $\H_{\CMS_0}^\infty$, \textit{Q.~E.~D.}

\paragraphtitle{Proof of the second lemma} We will show first that, for any $n \in \IN$, the subspace $\H_{\CMS_0}^n$ is the eigenspace of $\op P_{\CMS_0}^{(n)}$ for the eigenvalue 1.
This is obvious for $n = 0$, where we have $\op P_{\CMS_0}^{(0)} = \op \Pi_{\CMS_0}$ and $\H_{\CMS_0}^0 = \H_{\CMS_0}$.
Let us prove first, by recurrence on $n \in \IN$, that the subspace $\H_{\CMS_0}^n$ is included in the eigenspace of $\op P_{\CMS_0}^{(n)}$ for the eigenvalue 1.
As we've just seen, this assertion holds for $n = 0$.
Let us suppose that the assertion is proved for a given $n \in \IN$, and let us consider a quantum state $\ket \Psi \in \H_{\CMS_0}^{n+1}$.
We have per definition:
\begin{equation*}
\op P_{\CMS_0}^{(n+1)} \ket \Psi = \sum_{\cms_0 \in \CMS_0} \dotsc \sum_{\cms_n \in \CMS_0} \op \Pi_{\cms_0} \dop U_\tau \dotsm \op \Pi_{\cms_n} \dop U_\tau \op \Pi_{\CMS_0} \op U_\tau \op \Pi_{\cms_n} \dotsm \op U_\tau \op \Pi_{\cms_0} \ket \Psi
\end{equation*}
Since $\ket \Psi \in \H_{\CMS_0}^{n+1}$, the quantum state $\op U_\tau \op \Pi_{\cms_n} \dotsm \op U_\tau \op \Pi_{\cms_0} \ket \Psi$ belongs to $\H_{\CMS_0}$, and since the elementary evolution operator $\op U_\tau$ is unitary and $\op \Pi_{\cms_n}$ is a projection operator, we have $\op \Pi_{\cms_n} \dop U_\tau \op U_\tau \op \Pi_{\cms_n} = \op \Pi_{\cms_n}$, so that:
\begin{equation*}
\op P_{\CMS_0}^{(n+1)} \ket \Psi = \op P_{\CMS_0}^{(n)} \ket \Psi
\end{equation*}
Now the recurrence hypothesis yields $\op P_{\CMS_0}^{(n)} \ket \Psi = \ket \Psi$, so that $\op P_{\CMS_0}^{(n+1)} \ket \Psi = \ket \Psi$, which proves the recurrence.

\paragraphtitle{§} Let us prove now, for any $n \in \IN$, the inverse inclusion of the eigenspace of $\op P_{\CMS_0}^{(n)}$ for the eigenvalue 1 in the subspace $\H_{\CMS_0}^n$.
Since $\op P_{\CMS_0}^{(n)} \ket \Psi = \ket \Psi$ implies $\bra \Psi \op P_{\CMS_0}^{(n)} \ket \Psi = \braket \Psi \Psi$ for any quantum state $\ket \Psi \in \H$, it is sufficient to prove, by recurrence on $n \in \IN$, that $\bra \Psi \op P_{\CMS_0}^{(n)} \ket \Psi = \braket \Psi \Psi$ implies $\ket \Psi \in \H_{\CMS_0}^n$ for any quantum state $\ket \Psi \in \H$.
We have already proved this for $n = 0$, so let us assume that the assertion holds for a given $n \in \IN$.
We have per definition:
\begin{eqnarray*}
\bra \Psi \op P_{\CMS_0}^{(n+1)} \ket \Psi & = & \sum_{\cms_0 \in \CMS_0} \dotsc \sum_{\cms_n \in \CMS_0} \left\| \op \Pi_{\CMS_0} \op U_\tau \op \Pi_{\cms_n} \dotsm \op U_\tau \op \Pi_{\cms_0} \ket \Psi \right\|^2 \\
& = & \sum_{\cms_0 \in \CMS_0} \dotsc \sum_{\cms_n \in \CMS_0} \sum_{\cms \in \CMS_0} \left\| \op U_\tau \op \Pi_\cms \op U_\tau \op \Pi_{\cms_n} \dotsm \op U_\tau \op \Pi_{\cms_0} \ket \Psi \right\|^2
\end{eqnarray*}
Now for any quantum state $\ket \Phi \in \H$, $\sum_{\cms \in \CMS_0} \| \op U_\tau \op \Pi_\cms \ket \Phi \|^2 \leq \braket \Phi \Phi$, the case of an equality happening if and only if $\ket \Phi \in \H_{\CMS_0}$.
The sequence $(\bra \Psi \op P_{\CMS_0}^{(n)} \ket \Psi)$ is therefore decreasing, with an initial value $\bra \Psi \op P_{\CMS_0}^{(0)} \ket \Psi = \bra \Psi \op \Pi_{\CMS_0} \ket \Psi$, and we have $\bra \Psi \op P_{\CMS_0}^{(n+1)} \ket \Psi = \bra \Psi \op P_{\CMS_0}^{(n)} \ket \Psi$ if and only if, for all $\cms_0, \dotsc, \cms_n \in \CMS_0$, $\op U_\tau \op \Pi_{\cms_n} \dotsm \op U_\tau \op \Pi_{\cms_0} \ket \Psi \in \H_{\CMS_0}$.
Let us suppose now that we have $\bra \Psi \op P_{\CMS_0}^{(n+1)} \ket \Psi = \braket \Psi \Psi$.
This implies, because of the decrease of the sequence and of its initial value, that $\ket \Psi \in \H_{\CMS_0}$ and that $\bra \Psi \op P_{\CMS_0}^{(n)} \ket \Psi = \braket \Psi \Psi$.
We have thus $\bra \Psi \op P_{\CMS_0}^{(n+1)} \ket \Psi = \bra \Psi \op P_{\CMS_0}^{(n)} \ket \Psi$, which implies, as we have just seen, that for all $\cms_0, \dotsc, \cms_n \in \CMS_0$, $\op U_\tau \op \Pi_{\cms_n} \dotsm \op U_\tau \op \Pi_{\cms_0} \ket \Psi \in \H_{\CMS_0}$.
The recurrence hypothesis yielding $\ket \Psi \in \H_{\CMS_0}^n$, we have therefore $\ket \Psi \in \H_{\CMS_0}^{n+1}$, which proves the recurrence.
The eigenspace of $\op P_{\CMS_0}^{(n)}$ for the eigenvalue 1 is therefore $\H_{\CMS_0}^n$ for all $n \in \IN$.

\paragraphtitle{§} Now the subspace $\H_{\CMS_0}$ being finite dimensional, there must exist a finite number of distinct nested subspaces $\H_{\CMS_0}^n$, so that there exists an integer $n_{\CMS_0}$ such that:
\begin{equation*}
n_{\CMS_0} \eqdef \min \{ n \in \IN\ |\ \H_{\CMS_0}^n = \H_{\CMS_0}^\infty \}
\end{equation*}
The eigenspace of $\op P_{\CMS_0}^{(n)}$ for the eigenvalue 1 is therefore $\H_{\CMS_0}^\infty$ for any $n \geq n_{\CMS_0}$.
In the special case where $n_{\CMS_0} = 0$, we have $\op P_{\CMS_0}^{(n)} = \op \Pi_{\H_{\CMS_0}^\infty}$ for all $n \in \IN$, so that the sequence $(\op P_{\CMS_0}^{(n)})$ trivially converges towards $\op \Pi_{\H_{\CMS_0}^\infty}$.
Let us assume from now on that we have $n_{\CMS_0} \geq 1$.
Since $\op P_{\CMS_0}^{(n)}$ induces an hermitian operator on $\H_{\CMS_0}$, it induces also, for $n \geq n_{\CMS_0}$, an hermitian operator on the supplementary subspace $\H_{\CMS_0}^{\infty \perp}$ of its eigenspace $\H_{\CMS_0}^\infty$.
It is diagonalizable on an orthogonal basis of $\H_{\CMS_0}^{\infty \perp}$, with a finite number of real eigenvalues, which are all lying in the interval $[0, 1[$.
There exists therefore a number $p_{\CMS_0} \in [0, 1[$ such that $p_{\CMS_0}^{n_{\CMS_0}}$ be the greatest eigenvalue of this operator, which can be defined by:
\begin{equation*}
p_{\CMS_0}^{n_{\CMS_0}} \eqdef \max \{ \bra \Psi \op P_{\CMS_0}^{(n_{\CMS_0})} \ket \Psi / \braket \Psi \Psi \ |\ \ket \Psi \in \H_{\CMS_0}^{\infty \perp} \setminus \{ 0 \} \}
\end{equation*}

\paragraphtitle{§} For any initial quantum state $\ket \Psi \in \H_{\CMS_0}^{\infty \perp}$, $\bra \Psi \op P_{\CMS_0}^{(n_{\CMS_0})} \ket \Psi / \braket \Psi \Psi$ represents the probability that the quantum state be in $\H_{\CMS_0}$ at all times 0, ..., $n_{\CMS_0}$.
Now, as we have seen in the first lemma, the quantum state at time $n_{\CMS_0}$ will still be orthogonal to $\H_{\CMS_0}^\infty$, since it is of the form $\op \Pi_{\cms_{n_{\CMS_0}}} \op U_\tau \dotsm \op U_\tau \op \Pi_{\cms_0} \ket \Psi$ for a given sequence of mental states $\cms_0, \dotsc, \cms_{n_{\CMS_0}} \in \CMS_0$.
It belongs therefore to $\H_{\CMS_0}^{\infty \perp}$, so that the probability that it stays in $\H_{\CMS_0}$ for another $n_{\CMS_0}$ time steps is at most $p_{\CMS_0}^{n_{\CMS_0}}$, too.
We have therefore:
\begin{equation*}
\bra \Psi \op P_{\CMS_0}^{(2 n_{\CMS_0})} \ket \Psi / \braket \Psi \Psi \leq p_{\CMS_0}^{2 n_{\CMS_0}}
\end{equation*}
As we have seen, $\bra \Psi \op P_{\CMS_0}^{(t)} \ket \Psi / \braket \Psi \Psi$ is a decreasing function of $t$, so that we have more generally, for any $t \in \IN$:
\begin{equation*}
\bra \Psi \op P_{\CMS_0}^{(t)} \ket \Psi / \braket \Psi \Psi \leq p_{\CMS_0}^{n_{\CMS_0} \lfloor t / n_{\CMS_0} \rfloor}
\end{equation*}
where $\lfloor \cdot \rfloor$ denotes the floor function.

\paragraphtitle{§} Let $(\ket {\Psi_i})$ be an orthonormal basis of the subspace $\H_{\CMS_0}^{\infty \perp}$.
The partial trace of the operator $\op P_{\CMS_0}^{(t)}$ on this subspace is given, for any $t \in \IN$, by:
\begin{equation*}
\mathrm{Tr}_{\H_{\CMS_0}^{\infty \perp}} \op P_{\CMS_0}^{(t)} = \sum_i \bra {\Psi_i} \op P_{\CMS_0}^{(t)} \ket {\Psi_i}
\end{equation*}
so that we have:
\begin{equation*}
\mathrm{Tr}_{\H_{\CMS_0}^{\infty \perp}} \op P_{\CMS_0}^{(t)} \leq p_{\CMS_0}^{n_{\CMS_0} \lfloor t / n_{\CMS_0}  \rfloor} \dim \H_{\CMS_0}^{\infty \perp} \xrightarrow{t \to \infty} 0
\end{equation*}
Now this partial trace is the sum of the eigenvalues multiplied by the dimension of the respective eigenspace, and all these eigenvalues are positive.
The greatest eigenvalue of $\op P_{\CMS_0}^{(t)}$ on $\H_{\CMS_0}^{\infty \perp}$ converges therefore towards 0, too, which proves that the sequence of operators $(\op P_{\CMS_0}^{(t)})$ converges towards 0 on this subspace.

\paragraphtitle{§} We have already seen the eigenspace of $\op P_{\CMS_0}^{(t)}$ for the eigenvalue 1 is $\H_{\CMS_0}^\infty$ for any $t \geq n_{\CMS_0}$, and that the operator induced on $\H_{\CMS_0}^{\infty \perp}$ converges towards 0.
Furthermore, for any quantum state $\ket \Psi$ orthogonal to the subspace $\H_{\CMS_0}$, we have $\op P_{\CMS_0}^{(t)} \ket \Psi = 0$.
The sequence of operators $(\op P_{\CMS_0}^{(t)})$ converges therefore towards $\op \Pi_{\H_{\CMS_0}^\infty}$, \textit{Q.~E.~D.}

\paragraphtitle{Proof of the theorem} An initial mental state $\cms_i \in \CMS$ and an initial quantum state $\ket \Psi \in \H_{\cms_i} \setminus \{0\}$ being given, the probability $p_1$ that the mental state $\cms_i$ be experienced again at the next time step is given by:
\begin{equation*}
p_1 = \bra \Psi \dop U_\tau \op \Pi_{\cms_i} \op U_\tau \ket \Psi / \braket \Psi \Psi
\end{equation*}
If the mental state experienced at the next time step is different from $\cms_i$, which happens with a probability $1 - p_1$, then this mental state belongs to $\CMS_0 \eqdef \CMS \setminus \{ \cms_i \}$.
More generally, for any $t \in \IN^*$, the probability $p_t$ that the mental state $\cms_i$ have been experienced again at any time $t' \leq t$ is given by:
\begin{equation*}
p_t = 1 - \bra \Psi \dop U_\tau \op P_{\CMS_0}^{(t-1)} \op U_\tau \ket \Psi / \braket \Psi \Psi
\end{equation*}
The second lemma yields:
\begin{equation*}
p_t \xrightarrow{t \to \infty} 1 - \bra \Psi \dop U_\tau \op \Pi_{\H_{\CMS_0}^\infty} \op U_\tau \ket \Psi / \braket \Psi \Psi
\end{equation*}
and since the initial quantum state $\ket \Psi$, belonging to $\H_{\cms_i}$, is orthogonal to $\H_{\CMS_0}^\infty$, the first lemma yields:
\begin{equation*}
p_t \xrightarrow{t \to \infty} 1
\end{equation*}
The initial mental state $\cms_i$ will therefore be experienced again in the future with certainty,  \textit{Q.~E.~D.}

\paragraphtitle{Remark} This proof of the reincarnation theorem relies on the hypothesis that the Hilbert space be finite dimensional.
This assumption could be relaxed a little and reduced to the hypothesis that the set of all mental states accessible from the initial quantum state -- that it, the last mental states $\cms_n$ of sequences $\cms_0, \dotsc, \cms_n$ such that $\op \Pi_{\cms_n} \op U_\tau \dotsm \op \Pi_{\cms_0} \op U_\tau \ket \Psi \neq 0$ -- be finite, and the corresponding mental subspaces finite dimensional.

\paragraphtitle{Commentaries} Once again, we are addressing here philosophical, or rather spiritual questions which could not even be expressed in the frame of any scientific theory so far.
The reincarnation theorem has been derived with the greatest possible formal rigor, so that it is firmly established upon its foundations, which are compatible with everything we know about quantum physics today.
So rejecting without any further discussion the notion of ``reincarnation'' considered in this theorem -- the recurrence of mental states -- in no scientifically valid option any more.
But it is important in the first place to clarify what the theorem exactly means in order to avoid misunderstandings.
In a monist framework, contrary to the dualist conceptions of most religions supporting the idea of reincarnation, there is no mental entity like a soul which could be reincarnated in another body after the death of the last one.
The only thing that can be ``incarnated'' are mental states, \textit{i.e.} the totality of all subjective experiences taking place at a given instant.
And a mental state is being experienced whenever the quantum state, representing the state of affairs in the whole universe at the material level, becomes projected into the corresponding mental subspace, in which there is exactly one brain presenting the corresponding activity pattern for each subjective experience in the mental state.
This projection, or ``collapse'', is a stochastic process following the Born rule, which has been validated uncountable times by all quantum experiments.
Now the theorem states that, once a mental state has been experienced, the probability that, in the natural evolution of the physical state via physical interactions and collapse processes, this mental state be never experienced again, vanishes in the limit of an infinitely long time.
So if we take for granted the fact that the physical state will keep evolving forever the way it does today, -- and in particular if there is no sudden and unpredictable ``end of the world'' standing before us, -- it is certain that a mental state that has been experienced once will be experienced again in the future.
This is all the reincarnation theorem is stating; it is a mere recurrence theorem for mental states, quite similar in this respect to the Poincaré recurrence theorem for the state of classical systems in Hamiltonian dynamics.
But it is not a recurrence theorem for quantum states: The same mental state can be experienced in an infinite number of different quantum states belonging to the corresponding mental subspace, and even quantum states that differ very slightly from another can yield to very different evolutions on the long term, as we know from quantum chaos theory.

Now it follows trivially from the reincarnation theorem that every single subjective experience, once it has taken place for the first time, will be repeated again and again an infinite number of times.
Every single instant of your own mental life, in particular, will be experienced again and again in an infinite number of lives.
This is the point where we are getting very near to usual notions of reincarnation.
If you identify yourself with your subjective experiences, then you can say that you will be reincarnated in an infinite number of lives.
These lives might be very different from another; not all of the subjective experiences of your current life must take place again in each ``reincarnation'', nor must they occur in the same order, and your subjective experiences might even ``reincarnate'' in several lives taking place simultaneously (which approaches the notion of ``avatars'').
Now this notion of reincarnation has nothing supernatural; it happens, so to say, at random, whenever the physical evolution of the universe produces again a brain with an activity pattern corresponding to a subjective experience that has already taken place before.
It differs in this regard from the Buddhist conception of reincarnation, for instance, where the soul continues after death its journey on Earth by reincarnating in the body of another, possibly very different living being.
Actually, an important part of your subjective experiences probably won't be ``reincarnated'' on this Earth, because they contain representations of contingent elements of reality -- like the last breaking news, technological artifacts, mode accessories -- that most probably won't occur again in the culture history of Manhood.
But still they will be ``incarnated'' again -- on another planet quite similar to Earth.
And even your vision of the constellations in the night sky will be ``reincarnated'' once -- in another galaxy quite similar to the Milky Way.
Now how is this possible? The latest astronomical observations, interpreted in the frame of the  $\Lambda$-Cold Dark Matter model (a refinement of the Friedmann–Lemaître–Robertson–Walker general relativistic, homogeneous, isotropic model of the universe~\cite{Friedman1922}), suggest that the universe is going to keep expanding at a constant rate, the subsequent dilution of matter preventing on the long term the formation of new galaxies and stars, which will eventually all get extincted.
This is the so-called ``cold death'' scenario in cosmology.
Why do we seem to escape this scenario in lattice quantum field theory? This is definitely not related to the details of the physical interactions, and in particular of the choice of the graviton model we could use, since the reincarnation theorem doesn't depend on the exact form of the elementary evolution operator $\op U_\tau$.
The situation would certainly look different if the elementary evolution operator would change across time, which would probably be the case if we computed a semi-classical gravitational background at each time step, for instance.
But for now, with a constant elementary evolution operator, the model cannot account for an expansion of space-time; the galaxies, if they are drifting apart consequently to a ``big bang'' event, would meet again after having traveled through half of the universe because of its toroidal character.
This would be king of a ``big crunch'' event (but not due to a contraction of space-time) which would be followed by another ``big bang'' event that could possibly yield to the formation of another Milky Way and of another Earth on which your subjective experiences could be ``reincarnated''.
This conception is quite similar to the Buddhist cosmology, in which the world is supposed to be periodically destructed and re-created.

There are several ways of questioning the reality of this ``reincarnation'' in a purely scientific approach.
The reincarnation theorem relies on the finite dimensionality of the Hilbert space of quantum states, and thus on the assumption that the physical space itself is discrete and finite, which has elusive, but in principle measurable consequences.
Observing experimentally structures in differential scattering cross-sections revealing the finiteness of the lattice step or the quantization of momentum, or observing patterns in the angular distribution of the cosmic microwave background radiation revealing the toroidal character of space, would provide hints in favor of this model and thus of the reincarnation theorem, while their non-observation would put experimental constraints on its parameter space.
On the other hand, in a theoretical approach, fulfilling the program of constructive quantum field theory -- constructing a well-defined theory of interacting fields, compatible with the Standard Model, on the four dimensional Minkowski space-time -- would provide an alternative to our model where the Hilbert space presumably wouldn't be finite dimensional, so that the reincarnation theorem probably wouldn't hold.
Finally, on a conceptual level, developing an alternative model of the mental world or of the mind-body relationship within the frame of quantum theory would be a promising option too, since this is a completely new field of science.
There are probably plenty of interesting models to investigate once one has accepted to put subjective experience in an equation in the first place.

%TODO: Philosophical issues: Thought experiments
%\chapter{Thought experiments}

%TODO: Philosophical issues: Double-slit experiment
%\section{Double-slit experiment}

%TODO: Philosophical issues: Schrödinger's cat
%\section{Schrödinger's cat}

%TODO: Philosophical issues: Wigner's friend
%\section{Wigner's friend}

%TODO: Philosophical issues: EPR paradox
%\section{EPR paradox}
