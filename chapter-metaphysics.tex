\chapter{Metaphysics}
\label{Metaphysics}

\renewcommand{\epigraphwidth}{6cm}
\epigraph{As I have said so many times,\\God doesn't play dice with the world.}{Albert Einstein, \\ in \textit{Einstein and the Poet}~\cite{Hermanns1983}}

\section{Spinoza's philosophy}

Since the interpretation of Quantum Field Theory I am about to give has been inspired by Spinoza's classical work \textit{The Ethics}~\cite{Spinoza1677}, I shall make here a short presentation of its basic ideas. According to the causalist world view of classical mechanics, each individual existent thing -- an object, a thought -- has necessarily a cause which explains its existence at a given moment. These things are considered to be alterations, or ``modes'', of some fundamental ``substance'' constitutive of Nature as a whole. Since this substance has some of the fundamental properties attributed to God by Judaic theology -- self-caused, free, eternal, infinite (\textit{i.e.} containing everything) --, it has been identified by Spinoza to God itself, confounding thus the concept of `God' with what philosophers traditionally call `Nature'. The human intellect conceives the substance, as well as every individual existent thing, under the two aspects, or ``attributes'', of an extended (material) and of a thinking (mental) thing. This categorization, however, is nothing but a property of the human intellect and not an intrinsic property of the things themselves. Considered under its material aspect, a human being, for instance, consists in a body extending in the substance, \textit{i.e.} in God, whereas it consists in a mind thinking in God when considered under its mental aspect. Nevertheless, both are one and the same thing, so that the laws of Physics -- considered to be part of the nature of God -- could determine the laws of Psychology. The knowledge of God, which also encompasses the knowledge of the world in general and of Man in particular, is therefore considered to be the mind's highest good.

\section{Quantum metaphysics}

Interestingly, Spinoza's metaphysical concepts can be identified quite straightforwardly with the fundamental notions of Quantum Field Theory, thus providing them with a naturalistic basis. On the other hand, Quantum Field Theory, generally considered to be counter-intuitive, paradoxical and hardly understandable, becomes grounded in a very classical philosophical tradition and should thus become accessible to a broader range of Science philosophers.

The states (modes) of God (the substance) are evidently identified, under their material aspect, with the quantum states $\ket \Psi$ of the universe (the elements of the Hilbert space $\H$), and, under their mental aspect, with the mental states $\cms$ (the elements of $\CMS$). The relation between the material and the mental aspects is given by the decomposition $\H = \bigoplus_{\cms}^\perp \H_\cms$ of the Hilbert space, or equivalently by the orthogonal projection operators $\op{\Pi}_\cms$, relating each mental state $\cms$ with the set of all corresponding quantum states $\H_\cms$. Furthermore, the nature of God encompasses the laws of Physics, given by the Hamilton operator $\Hop$, or more precisely by the elementary evolution operator $\op U_\tau \eqdef \exp{- \i 2 \pi \Hop \tau / \h}$. God can finally be defined as a mathematical structure $\God$ given by:
\begin{equation*}
\God \eqdef \left( \H \times \CMS, (\op{\Pi}_\cms), \op U_\tau \right)
\end{equation*}
The states of God, taking the general form $\GodState{\ket \Psi}{\cms}$, are said to be `real' if $\ket \Psi \in \H_\cms \setminus \{ 0 \}$ and `virtual' otherwise. By extension, we will say that a quantum state $\ket \Psi \neq 0$ is `real' if it belongs to one of the subspaces $\H_\cms$. An elementary evolution step of the state of God proceeds from a real state $\GodState{\ket \Psi_0}{\cms_0}$, first evolving to a generally virtual state $\GodState{\op U_\tau \ket{\Psi_0}}{\cms_0}$ and eventually collapsing to one of the real states $\GodState{\op{\Pi}_{\cms_1} \op U_\tau \ket{\Psi_0}}{\cms_1}$ with a probability $\bra{\Psi_0} \dop U_\tau \op{\Pi}_{\cms_1} \op U_\tau \ket{\Psi_0} / \braket{\Psi_0}{\Psi_0}$.
