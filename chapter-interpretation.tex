\chapter{Interpretation}
\label{Interpretation}

\section{The role of consciousness}

As it results from the preceding description of the processes taking place in the evolution of the state of God, not only the purely material processes described by the Hamiltonian evolution operator $\op U_\tau$, but also the mental processes described by the collapse operators $(\op{\Pi}_\cms)$ play a central role in the evolution of the quantum state $\ket \Psi$ of the universe. In the following, we will call `consciousness state' any collective mind state $\cms$ and `consciousness' the phenomenon of experiencing it. This phenomenon must be very carefully distinguished from the purely material processes of conscious thinking happening at the neural level within brains, although both are closely related to each other.

Quantum phenomena, like the superposition of an atom in an excited and a decayed state, or the superposition of a photodetector in an activated and an unactivated state -- as in the quantum measurement example given in section \ref{Quantum measurement} --, have always proved to be very puzzling to us, because they show that quantum processes don't fit within our mental categories, in which a photodetector must be either activated or not, for instance. This is not a result of a limitation of our intelligence that we could overcome by learning to think in a new way corresponding more adequately to the physical reality. No, at a very fundamental level, there isn't and there will never be any individual mind state $\ims$ corresponding to the superposition of a brain having ``seen'' a photodetector both activated and not, although this superposition does exist at the material level. This inadequacy between our mental categories and the material reality is a matter of fact having profound consequences for the process of consciousness. The contents of a consciousness state $\cms$ cannot simply be a representation of the material reality (even a partial one), because material reality explores possibilities going far beyond the realm of what we can grasp with our mental categories. An arbitrary consciousness state $\cms$ can only ``match'' more or less good the current quantum state $\ket \Psi$ of the universe, the number $\bra \Psi \op{\Pi}_\cms \ket \Psi / \braket \Psi \Psi$, lying between 0 and 1, measuring how good the fit is. If it equals 1, the fit is perfect (although $\cms$ remains a partial representation of the quantum state $\ket \Psi$) and $\cms$ is being experienced with certainty. If it equals 0, there is no fit and $\cms$ cannot be experienced. If it lies inbetween, any other consciousness state $\cms$ could be experienced too, the numbers $\bra \Psi \op{\Pi}_\cms \ket \Psi / \braket \Psi \Psi$ defining the probability law according to which the actually experienced consciousness state will be selected. So the quantum state $\ket \Psi$ of the universe determines the contents of the experienced consciousness state $\cms$, according to a probability law reflecting how good the mental categories in $\cms$ fit the material reality $\ket \Psi$. On the other hand, and this is probably even more astonishing, the experience of the consciousness state $\cms$ will reduce the quantum state $\ket \Psi$ of the universe to its component $\op{\Pi}_\cms \ket \Psi$ corresponding to the mental categories in $\cms$. So we can say that consciousness actively shapes the material reality according to its own mental categories -- or more poetically, that you are putting human order into the world with every glance you take at it!

In the quantum measurement example given in section \ref{Quantum measurement}, for instance, the quantum superposition of two states of the brain of the observer, having either observed the photodetector activated or not, resolves to one of the components corresponding to the mental categories `I have seen the photodetector activated' and `I haven't seen it activated yet'. Because the quantum state of the rest of the universe (here specifically the photodetector and the decaying atom) is correlated to the quantum state of the brain via sensory perception, this reduction of the quantum state of the universe to a given mental category will have consequences for the rest of the universe, too. For instance, if the observer makes the conscious experience of seeing the photodetector activated, the state of the photodetector will also reduce to the activated state only, because the unactivated state is only correlated to the component of the state of the brain corresponding to the mental category `I haven't seen the photodetector activated yet', which is being dropped. So the mental categories, which only concern the quantum state of the brain originally, get transposed to external objects via the correlation induced by sensory perception between them and the brain in the quantum state $\ket \Psi$ of the universe. Similarly, the state of the decaying atom will also reduce to the decayed state because the non-decayed state is only correlated to the component of the state of the brain corresponding to the mental category `I haven't seen the photodetector activated yet', which is being dropped.

If the observer had made instead another kind of measurement on the decaying atom, e.g. measuring its position (which we suppose here to be uncorrelated with its decay), the quantum state of the atom would have been reduced accordingly, so that it would only be present in the region of space where it has been observed, whereas the decayed and non-decayed states would still remain in a quantum superposition, since they are not correlated with the consciousness state of the observer. So the way our mental categories get transposed to external objects strongly depends on the way we are observing them, \textit{i.e.} on the way we are letting them get correlated to the quantum state of our brains.

\section{Epistemological considerations}

From a historical perspective, it should become quite clear today why the founders of Quantum Physics have had such difficulties to agree on an interpretation of this new branch of Physics. Starting with a few subatomic experiments, like the measurement of the emission spectrum of hydrogen atoms, they ended up with a theory bringing along a twofold scientific revolution and profoundly revising our world view.

The first revolution, which is nowadays widely accepted, concerns the fact that material reality cannot be described within our usual mental categories. The most classical example is the so-called wave-particle duality, which implies that elementary particles, and as a consequence also atoms and molecules for instance, can occupy several positions in space at a time and that their motion follow wave equations and interference patterns typical for wave phenomena. And ultimately, not only invisible particles, but also configurations of the whole universe can combine with each other via wave amplitudes and interfere in their evolution in a similar way as waves would do. This has been a big paradigm change compared to the ideal of Classical Physics, where intelligibility, \textit{i.e.} the adequation to our mental categories, was considered an essential characteristic of any scientific theory. The position of Louis de Broglie, for instance, is typical for the efforts to resist this paradigm change. After having proposed himself the relation $\lambda = \h / p$ between the momentum $p$ of a particle and the wavelength $\lambda$ of the corresponding wave phenomena, he developed the Pilot-Wave Theory, an alternative interpretation of Quantum Mechanics (that we know today to be false) according to which both the particle and the corresponding wave have their own existence and can be described as in Classical Physics, \textit{i.e.} according to our mental categories -- the particle having a definite trajectory and being ``guided'' by the accompanying wave.

The second scientific revolution, which is far from being over yet, concerns the fact that consciousness actively modifies the quantum state of the universe, according to its own mental categories and in its own way, which cannot be reduced to other, purely material phenomena: Technically, the collapse of a quantum state from $\ket \Psi$ to $\op{\Pi}_\cms \ket \Psi$ obviously cannot be reduced to a Hamiltonian evolution of the form $\op U_\tau \ket \Psi$. This is of course a radical paradigm change compared to the Cartesianism of the Copenhagen interpretation, according to which the consciousness of the ``observer'' passively takes notice of some aspects of the material world, e.g. the (supposedly well-defined) state of a measurement apparatus. A typical opponent to this paradigm change is Albert Einstein, who saw with very critical eyes the ``spooky action at a distance'' implied by the collapse of the quantum state of the universe, \textit{i.e.} the instantaneous modification of the quantum state of a distant object happening when the consciousness state of a previously correlated brain is being selected. The famous Einstein-Podolsky-Rosen thought experiment, which has been conceived to illustrate these non-local features of collapse (\textit{i.e.} its incompatibility to one of the central principles of Special Relativity), was thought to invalidate definitely the hypothesis of collapse, because it was conceived under the assumptions that all physical phenomena should obey the same laws, described in Quantum Theory by the Hamiltonian evolution operator $\op U_\tau$, and that collapse should ultimately be described in that way instead of using the ad-hoc assumption of the intervention of a projection operator $\op{\Pi}_\cms$. However, as this experiment has been realized for instance by the team of Alain Aspect~\cite{Aspect1982} in experimental conditions becoming more and more sophisticated, the non-locality of quantum measurement has always been demonstrated very clearly, so that it makes no doubt today that collapse does obey other physical laws than Hamiltonian evolution alone.

The first milestone of this second scientific revolution has been set by John von Neumann with the idea that ``mind causes collapse''~\cite{Neumann1932}. This idea addresses a leak in the Copenhagen interpretation, where we distinguish between a ``macroscopic world'', supposed to be ruled by the laws of Classical Physics, and a ``microscopic world'', ruled by the laws of Quantum Physics. The interface between both worlds is build by measurement apparatuses, which are supposed to cause the collapse of the quantum state of the microscopic world when they interact with it. This interpretation relies on the assumption that no quantum phenomenon can be observed without the help of a measurement apparatus, which is obviously false. For instance, you can observe with your naked eyes the diffraction patterns of light passing in the dark through the fine structures of woven fabric, and that is a genuine quantum phenomenon. One could maybe ``save'' the Copenhagen interpretation by considering that the eye constitute the measurement apparatus in that case, but where do an eye actually begin: With the cornea, the pupil, the retina, the retinal ganglion cells? Or even inside the brain, after the neural processing of visual perceptions? Defining the frontier between the microscopic and the macroscopic world seems to be a rather arbitrary operation and it is therefore not really intellectually satisfying. The only thing we are sure of is that, ultimately, our consciousness ``resolves'' quantum superpositions according to our mental categories. This idea that, ultimately, consciousness causes collapse was once known as the `standard interpretation' of Quantum Mechanics. It has been almost forgotten since, perhaps because it had been originally formulated all to vaguely to be taken seriously. In this book, I am formulating it again using a very precise and well-defined formalism, so that one can unambiguously derive its implications on a very solid basis. I hope that this contribution will help reconsidering the profound implications of this second scientific revolution and widening its acceptance in the scientific community.

\section{The Spinozist approach to Quantum Physics}

The Spinozist aspects of our interpretation of Quantum Field Theory concern this second scientific revolution, \textit{i.e.} the role of consciousness in physical processes. Historically, Spinoza's philosophy developed on top of Cartesianism, which considers the material world to be a mechanical, deterministic one and consciousness to be a passive, external observer of the happenings in the material world; although this material world is supposed to obey the very strict laws of Classical Mechanics, the mental world is supposed to be absolutely free, independent of the material one and obeying no specific laws. Spinoza conserved this mechanical view of the material world, but tried to ground the mental world upon the material one, considering that consciousness cannot exist independently of a material body, that it reflects the state of its material substrate and therefore obeys the same laws, which can be transposed, in principle, to the mental world. Thus, consciousness becomes again part of Nature; it isn't considered any more to exist in an ideal realm exterior to the contingencies of the material world.

Our interpretation of Quantum Field Theory relates to the Copenhagen interpretation in a similar way as Spinozism relates to Cartesianism. In the Copenhagen interpretation, the microscopic world only -- and only the limited system under consideration -- is supposed to obey the laws of Quantum Physics, \textit{i.e.} the Hamiltonian evolution and the collapse as the system interacts with a measurement apparatus. On the contrary, the macroscopic world, including the observer, is supposed to exist in an ideal realm where only the (Cartesian) laws of Classical Physics apply. In a genuine Spinozist approach, our interpretation grounds this ideal realm upon the material realm of the quantum world: The arbitrary distinction between a microscopic and a macroscopic world vanishes, whereas collapse is supposed not to happen in an interaction with a measurement apparatus -- a mere artifact -- but with brains -- the material substrate of a fundamental aspect of Nature, consciousness. Mind becomes thus again part of Nature, and comes along with its own properties and physical laws, completing the Hamiltonian evolution laws of purely material processes. Of course, the relation between mind and body is much more complex than in classical Spinozism, but I think that the basic approach of the problem of consciousness is essentially the same, so that we can say, in that sense, that we are developing here a Spinozist interpretation of Quantum Physics.
